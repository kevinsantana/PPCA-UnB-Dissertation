\documentclass[12pt]{article}

\usepackage{styles/sbc-template}
\usepackage{graphicx,url}
\usepackage[utf8]{inputenc}
\usepackage[brazil]{babel}

     
\sloppy

\title{Exploring Energy Flow Classifier to Identify \\ Fraudulent Cryptocurrency Transactions}

\author{Kevin S. Araujo\inst{1}, Rodrigo Bonifacio de Almeida\inst{1}, 
  Fabiano Cavalcanti Fernandes\inst{2} }


\address{Departamento de Ciências da Computação -- Universidade de Brasília (UnB) \\
  -- Campus Universitário Darcy Ribeiro, Brasília-DF
\nextinstitute
  Instituto Federal de Brasília (IFB) -- Taguating, DF -- Brazil
  \email{kevin.araujo@aluno.unb.br, rbonifacio@unb.br, fabiano.fernandes@ifb.edu.br}
}

\begin{document} 

\maketitle

\begin{abstract}
  This is a work in progress.
\end{abstract}

\section{Introduction} \label{sec:introduction}

In 2008, a global financial crisis in the real estate sector occurred \cite{bordo2008historical}, which was caused by
the State providing easy credit \cite{murphy2008analysis}. A cryptographic document began circulating on a mailing list
for cryptographers, signed by the pseudonymous Satoshi Nakamoto, which compiled detailed data and discoveries made
through cypherpunk innovations. Nakamoto utilized this knowledge to create an electronic transaction system that did not
require the involvement of a third-party moderator. In essence, Nakamoto's work involved using math, programming, and
cutting-edge cryptography to publish a map for removing governmental presence from financial transactions. The recent
collapse of the economy had demonstrated that governments cannot be trusted, and Nakamoto's solution was to create a
currency that was mathematically impossible to be corrupted --- Bitcoin \cite{nakamoto2008bitcoin}. Although Bitcoin does
present a solution to the corruptible nature of money, it does possess flaws and is target of more sophisticated frauds.

Detecting such sophisticated frauds requires advanced analytical techniques. Identifying anomalous patterns within complex
data streams, such as network traffic or financial transactions, remains a
critical challenge. Traditional methods often struggle with the high dimensionality, evolving nature, and sheer volume
of modern data. In the domain of Network Intrusion Detection (NID), flow-based analysis offers a valuable abstraction by
aggregating packet-level information into connection summaries, reducing data complexity while retaining essential
behavioral characteristics. Within this context, a promising approach grounded in principles from statistical physics
is the Energy-based Flow Classifier (EFC). Originally proposed using the Inverse Potts model, EFC characterizes the
probability distribution of normal network flows through an energy function derived from observed data patterns
\cite{pontes2019}. Configurations representing typical, legitimate flows are assigned low energy values by the model,
whereas anomalous or potentially malicious flows, deviating from the learned normality, manifest as high-energy states.
Subsequent research has further explored the capabilities of this energy-based framework, highlighting its potential for
open-set recognition the challenging task of identifying novel anomalies not encountered during the model's training
phase \cite{souza2022novelopensetenergybased}. The fundamental principle of assigning an energy score as a measure of typicality provides a
robust and theoretically grounded mechanism for distinguishing normal system behavior from potentially illicit activities.

While Bitcoin was designed to circumvent traditional financial system vulnerabilities \cite{nakamoto2008bitcoin}, it is
not immune to manipulation and anomalous activities, necessitating robust detection mechanisms \cite{zhang2020financial,
zainal2018review}. Addressing this challenge within the Bitcoin ecosystem, this paper investigates the application of the
Energy-based Flow Classifier (EFC), a technique adept at identifying deviations from established norms in complex data
\cite{pontes2019, souza2022novelopensetenergybased}. Leveraging its foundation in statistical physics, EFC quantifies the typicality of
data points through an energy score, whereby normal, expected transaction patterns correspond to low energy states, and
significant deviations—potentially indicative of fraud or manipulation—manifest as high-energy anomalies. The central
objective of this work is, therefore, to explore the efficacy of EFC in identifying such anomalous operations within a
real-world Bitcoin transaction dataset, assessing its potential as a tool for enhancing the security and integrity of
cryptocurrency exchanges.


\bibliographystyle{sbc}
\bibliography{bibliography/sbc-template.bib}

\end{document}
