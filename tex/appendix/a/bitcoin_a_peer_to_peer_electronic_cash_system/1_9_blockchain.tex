\subsection{Blockchain}
A distributed ledger system is composed of multiple nodes that broadcast transactions. To ensure consensus on the
correct ledger, it is necessary to develop a mechanism that allows all nodes to agree on the validity of each
transaction~\cite{el2018review}.

The core concept of the original Bitcoin paper~\cite{nakamoto2008bitcoin} is based on the assumption that a distributed
ledger will be trusted if it has been subject to a large amount of computational effort. This idea is implemented
through the use of the many-zeroes game, which involves proving that a particular block in the chain contains a hash
that is difficult to reverse-engineer.

Rather than hashing the entire ledger repeatedly, it is more efficient to allow for the accumulation of computational
effort over time. Transactions are grouped into blocks and added to the chain in a linear fashion, with each new block
containing a reference to the previous one. This approach allows for the creation of a tamper-evident history of
transactions that is trusted by network participants due to the large amount of computational work required to
manipulate it.

%% figure 1.11
\figuraBib{img/chapter-2/2_2_9_blockchain/11_blocks.png}{Blocks on a
blockchain}{3blue1brown_blocks}{blocks}{width=.85\textwidth}%

The block is a collection of transactions enclosed with a unique identifier, known as proof-of-work (PoW), which serves
as evidence of the computational effort expended in validating the block. In PoW schemes, the miner must solve a complex
mathematical problem to validate the block and add it to the blockchain. The difficulty level of this problem is
determined by the target number of leading zeros required in the hash value of the block.

%% figure 1.12
\figuraBib{img/chapter-2/2_2_9_blockchain/12_block-ordering.png}{Because blocks are chained together like this, instead
of calling it a ledger, this is commonly called a
“blockchain”}{3blue1brown_blockchain}{blockchain}{width=.85\textwidth}%

A block is considered valid if it contains a proof-of-work (PoW) value, analogous to how a transaction is only
considered valid when signed by its sender. Additionally, maintaining the integrity of the blockchain requires that
blocks are not rearranged as this would disrupt the transaction history. To address this issue, each new block must
begin with the hash of the previous block (hash-based chain), ensuring that the order of the blocks remains consistent.

\subsubsection{Block Creators: Miners}
To maintain the integrity of our ledger after it has been split into blocks, we have introduced a new process for adding
new transactions. This involves grouping together transactions into blocks and computing a proof of work. As part of our
updated protocol, anyone in the world is allowed to act as a "block creator". The responsibility of the block creator is
to listen for broadcasted transactions, collect them into a block, and then perform a significant amount of
computational work to find a special number that will result in the hash of the block starting with 60 zeros. This
computed hash value is then broadcasted to the network as proof of work~\cite{wood2014ethereum}.

A special transaction can be included at the beginning of each block, where the creator is rewarded with a predetermined
amount of digital currency. This practice has been suggested as a means of compensating individuals for their efforts in
constructing blocks within a distributed ledger system~\cite{ding2020incentive}.

%% figure 1.13
\figuraBib{img/chapter-2/2_2_10_block_creators_miners/13_block-reward.png}{Block
reward}{3blue1brown_block_reward}{reward}{width=.45\textwidth}%

The block reward is a unique exception to our usual transaction acceptance rules in the Ledger Dollar economy, as it
does not require signature verification and increases the total number of currency units with each new block.

The process of creating blocks, known as "mining", involves a significant amount of work and introduces new currency
into the economy. However, when discussing miners, it is essential to understand that they are primarily focused on
listening to transactions, constructing blocks, broadcasting them, and receiving newly minted currency as a reward for
their efforts.

For miners, each block can be thought of as a miniature lottery where individuals guess numbers rapidly until one person
finds a combination that results in a hash starting with many zeros, earning the resulting reward. In contrast to
mining, non-mining Bitcoin users no longer need to record all individual transactions on their personal ledger. Instead,
they can simply monitor block production and rely on the fact that these blocks contain verified transactions. This
approach is more manageable than maintaining a comprehensive transaction ledger.

In the consensus algorithm used by Bitcoin and other cryptocurrencies, a mechanism known as the "longest chain rule" is
employed to resolve potential conflicts between competing blocks. Specifically, if two miners broadcast distinct
blockchains with conflicting transaction histories, the system defers to the one that has been the longest in terms of
cumulative proof-of-work effort expended on it, which is assumed to be more resistant to
manipulation~\cite{buterin2014next}. If there is a tie between two competing blocks, it may be necessary to wait for
additional information to determine which block is longer. This process relies on the assumption that the longest chain
represents the most widely accepted version of the blockchain. However, this approach has been subject to criticism due
to its reliance on proof-of-work mechanisms, which require significant computational effort and can lead to
centralization.

\subsubsection{Attempt Fraud On The Blockchain}

To evaluate the trustworthiness of this method, it is instructive to consider what steps an individual, such as Alice,
would need to take in order to deceive the system. In particular, suppose that Alice desires to purchase an item from
Bob for 100 Ledger Dollars (LD), but does not actually possess those LDs. She might attempt to send a block to Bob
containing a line indicating "Alice pays Bob 100 LD" without broadcasting this block to the broader network. By doing
so, Bob would believe that he had been paid and provide Alice with the item she desires. However, at a later time, Alice
could re-enter the economy and spend those same 100 LD elsewhere. When Bob attempts to spending those same 100 LD, other
individuals in the network may not recognize them as valid, leading to the potential for deception to be detected.

The process of creating a fraudulent transaction in a blockchain network requires a valid proof-of-work (PoW) that is
found before other miners who are listening to the same set of transactions as the attacker, each working on their own
block. This is a difficult task but can be accomplished if the attacker has a significant portion of the network's
computation power. If Alice is able to find the PoW before other miners, she can create a fraudulent transaction and
present it to Bob (but not to anyone else)~\cite{fang2022cryptocurrency}.

However, Bob will continue to receive broadcasts from other miners, and Alice did not inform these miners about the
block she produced for Bob. Therefore, they will not include this block in their own versions of the blockchain. As a
result, Bob will be hearing conflicting chains: one from Alice and another from everyone else~\cite{TAN2022101625}.
According to the protocol, Bob always trusts the longest chain he knows about, which may create challenges for detecting
and resolving fraudulent transactions in the network.

The probability of Alice's computational resources being smaller than the combined computational resources of the rest
of the network is high, and as a result, it is more likely for the rest of the network to find a valid proof of work for
their next block before she does. Additionally, if Alice has less than 50\% of the total computation on the network
(which is highly probable), she will outpace everyone else indefinitely will be nearly
impossible~\cite{nakamoto2008bitcoin}.

Eventually, when Alice fails to maintain her chain longer than the rest of the network, Bob will reject what he is
hearing from Alice and follow the longer chain that everyone else is working on. This is because creating blocks
requires significant computational effort, making it extremely difficult for any individual or group to manipulate the
consensus~\cite{szabo2005bit}.

It's worth noting that while building a single fraudulent block may be possible, maintaining the lie for an extended
period is challenging. Therefore, users should exercise caution and wait for several new blocks to be added on top of a
newly discovered block before trusting it as part of the main chain. By doing so, they can ensure that they are not
being tricked by a malicious actor attempting to manipulate the network \cite{dupont2019cryptocurrencies}.

%% figure 1.14
\figuraBib{img/chapter-2/2_2_11_attempt_fraud_on_the_blockchain/14_dont-trust-yet.png} {Blocks are most trustworthy when
they aren't brand new}{3blue1brown_trust}{trust}{width=.85\textwidth}%

\subsubsection{Ledger Dollars vs. Bitcoin}
The distributed ledger system based on proof-of-work, as demonstrated by Bitcoin and other cryptocurrencies, involves a
mining process where miners compete to solve a computational puzzle in order to validate transactions and add them to
the blockchain. This is accomplished through the use of hash functions, which are designed to be difficult to reverse
engineer, thereby ensuring the integrity of the distributed ledger \cite{bashir2017mastering}. The proof-of-work
challenge may involve finding a special number that will make the hash of the block start with 60 zeros. However, in
practice, this is achieved by systematically changing the number of zeros so that it takes approximately 10 minutes for
miners to find a new block ~\cite{nakamoto2008bitcoin}.

As a result of this process, a block reward is awarded to the miner who successfully validates a block. Initially, the
reward was set at 50 Bitcoin per block, but it has since been reduced to 6.25 Bitcoin per block every 210,000 blocks
~\cite{nakamoto2008bitcoin}. However, miners can also earn transaction fees by including them in the validation process
of transactions.

%% figure 1.15
\figuraBib{img/chapter-2/2_2_12_ledger_dollars_vs_bitcoin/15_limited-to-2400.png}{Transactions on a bitcoin blockchain
is limited}{3blue1brown_limited}{limited}{width=.85\textwidth}%

Considering Bitcoin's objective of approximately one block addition per 10 minutes, its processing capacity is
constrained to about 4 Bitcoin transactions per second, with some variability. By comparison, Visa handles an average of
approximately 1,700 transactions per second, with the capability to process over 24,000 per second. The relatively
slower processing speed of Bitcoin leads to higher transaction fees, as they determine the selection of transactions
included in new blocks by miners. Moreover, Bitcoin has faced criticism for its significant energy consumption. While
the proof-of-work concept effectively combats fraud, it necessitates an immense allocation of resources for block
mining.

According to the Cambridge Bitcoin Electricity Consumption Index, the present annual electricity consumption for Bitcoin
mining (as of 2021) is estimated at around 115 Terrawatt-Hours. To provide context, this consumption surpasses the
energy usage of the entire country of Finland. Since 2008, an alternative approach to proof of work, known as "proof of
stake," has emerged, offering a substantial reduction in energy requirements. Several newer cryptocurrencies have
embraced this methodology~\cite{CBECS2021}.

