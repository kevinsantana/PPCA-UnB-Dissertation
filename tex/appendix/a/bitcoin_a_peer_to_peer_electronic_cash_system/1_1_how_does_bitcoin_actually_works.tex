\subsection{How Does Bitcoin Actually Work?}
Bitcoin's creation was prompted by the need for a secure and decentralized system of transferring value. The solution to
this problem involved an intriguing mathematical puzzle that required the invention of new concepts such as digital
signatures and cryptographic hash functions.

Creating a new cryptocurrency is a complex process that involves several steps, including developing a consensus
mechanism, creating a blockchain, implementing security measures, and ensuring decentralization. To understand how
Bitcoin works and identify potential areas for design improvements, it can be helpful to examine the technical details
of its underlying protocols~\cite{nakamoto2008bitcoin}. Alternative cryptocurrencies have emerged as a result of
different design choices made by their creators, which has led to a diverse ecosystem of digital currencies with varying
features and use cases.

While the underlying technology may seem complex to some, it is important to note that using a cryptocurrency does not
require an in-depth understanding of its mechanics~\cite{barski2014bitcoin}. Just like swiping a credit card, users can
take advantage of user-friendly applications that enable seamless sending and receiving of these digital assets.

The concept of cryptocurrency revolves around enabling individuals to conduct transactions without relying on a
centralized entity for trust verification. Typically, when using a credit card to purchase goods or services, one must
rely on banks (or a network of banks) to correctly debit the user's account and credit the recipient's account. The
majority of currencies are issued by governments, which can exercise some level of control over their respective
currencies through means such as adjusting the money supply. As a result, holders of these currencies must place a
certain degree of trust in the government issuing them to manage them effectively.

The concept of Bitcoin was inspired by the desire to overcome the limitations of traditional financial systems.
According to Nakamoto (2008, p.1)~\cite{nakamoto2008bitcoin}:

\begin{displayquote}
    \textit{the root problem with conventional currencies is all the trust that's required to make it work}
\end{displayquote}

To address this issue, Bitcoin was designed as a decentralized digital currency that operates without a central
authority or intermediary. The money supply of Bitcoin is fixed and determined by its underlying algorithm, making it
resistant to inflation and manipulation. In addition, transactions in the Bitcoin network are recorded on a public
ledger called the blockchain, which ensures transparency and accountability. Bitcoin allows for direct peer-to-peer
payments without the need for intermediaries, such as banks or payment processors. This property of Bitcoin eliminates
the need for trust in a central authority and enables participants to transact with each other directly, thereby
reducing transaction costs and increasing efficiency.

The concept of decentralization in trustless payment systems has been subject to debate among readers. However, this
discussion is beyond the scope of our current topic. While personal needs for trustless payments may vary, the question
of whether such a system is technically feasible remains an intriguing one. Cryptography, which originated from
encrypting messages, employs deep mathematical concepts to achieve its objectives. The remarkable effectiveness of
cryptographic tools extends beyond confidential communication into other domains. For instance, the development of a
decentralized currency presents a significant challenge that can be addressed by applying cryptographic
techniques~\cite{diffie2022new}.

\subsubsection{Creating Your Own Cryptocurrency}
One common scenario where distributed ledgers can be useful is when multiple individuals frequently exchange small
amounts of money, such as paying for shared expenses like dinner bills. To simplify this process, they may choose to
maintain a communal ledger that records these transactions in a manner similar to using physical currency. By doing so,
participants can easily keep track of their contributions and settle up when necessary.

%% figure 1.1
\figuraBib{img/chapter-2/2_2_2_creating_your_own_cryptocurrency/1_ledger.png}{A ledger is a record of financial
transactions, utilized for monitoring the accounts of all parties
involved}{3blue1brown_ledger}{ledgerr}{width=.85\textwidth}%

The proposed ledger system would be a publicly accessible platform similar to a website where users can add new entries.
At the end of each month, participants could review the list of transactions and calculate the total sum. If an
individual has spent more than they have received, they would contribute that amount to the collective pool, while those
who have received more than they have spent would withdraw funds from the pool.

The protocol for participation in the system involves the following steps:

\begin{enumerate}
    \item Any individual can add entries to the distributed ledger;
    \item At the end of each month, all participants gather to reconcile their accounts using physical currency.
\end{enumerate}

However, a potential issue arises with a public ledger that allows any individual to add entries. How can one ensure
that Bob does not enter "Alice pays Bob 100" without Alice's approval? There is a Cryptography solution: \emph{Digital
signatures}.

