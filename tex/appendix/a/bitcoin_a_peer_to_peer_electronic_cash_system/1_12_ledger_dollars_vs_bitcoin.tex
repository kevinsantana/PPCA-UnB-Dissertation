\subsubsection{Ledger Dollars vs. Bitcoin}
The distributed ledger system based on proof-of-work, as demonstrated by Bitcoin and other cryptocurrencies, involves a
mining process where miners compete to solve a computational puzzle in order to validate transactions and add them to
the blockchain. This is accomplished through the use of hash functions, which are designed to be difficult to reverse
engineer, thereby ensuring the integrity of the distributed ledger \cite{bashir2017mastering}. The proof-of-work
challenge may involve finding a special number that will make the hash of the block start with 60 zeros. However, in
practice, this is achieved by systematically changing the number of zeros so that it takes approximately 10 minutes for
miners to find a new block ~\cite{nakamoto2008bitcoin}.

As a result of this process, a block reward is awarded to the miner who successfully validates a block. Initially, the
reward was set at 50 Bitcoin per block, but it has since been reduced to 6.25 Bitcoin per block every 210,000 blocks
~\cite{nakamoto2008bitcoin}. However, miners can also earn transaction fees by including them in the validation process
of transactions.

%% figure 1.15
\figuraBib{img/chapter-2/2_2_12_ledger_dollars_vs_bitcoin/15_limited-to-2400.png}{Transactions on a bitcoin blockchain
is limited}{3blue1brown_limited}{limited}{width=.85\textwidth}%

Considering Bitcoin's objective of approximately one block addition per 10 minutes, its processing capacity is
constrained to about 4 Bitcoin transactions per second, with some variability. By comparison, Visa handles an average of
approximately 1,700 transactions per second, with the capability to process over 24,000 per second. The relatively
slower processing speed of Bitcoin leads to higher transaction fees, as they determine the selection of transactions
included in new blocks by miners. Moreover, Bitcoin has faced criticism for its significant energy consumption. While
the proof-of-work concept effectively combats fraud, it necessitates an immense allocation of resources for block
mining.

According to the Cambridge Bitcoin Electricity Consumption Index, the present annual electricity consumption for Bitcoin
mining (as of 2021) is estimated at around 115 Terrawatt-Hours. To provide context, this consumption surpasses the
energy usage of the entire country of Finland. Since 2008, an alternative approach to proof of work, known as "proof of
stake," has emerged, offering a substantial reduction in energy requirements. Several newer cryptocurrencies have
embraced this methodology~\cite{CBECS2021}.
