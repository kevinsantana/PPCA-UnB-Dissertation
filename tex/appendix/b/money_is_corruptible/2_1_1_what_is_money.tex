\subsection{What is Money}
If we asked: \textit{What is man's greatest invention?} What would your answer be? There are a lot of options. Would it
be fire? Because it gives us warmth, protection, and the ability to cook our meals? Or perhaps you would pick the wheel?
Because it's the driving force being the beginnings of trade, commerce, and travel. While both of those are excellent
choices, most of the time when we think about the greatest inventions of mankind, we tend to forget one of the most
important ones of all: money. Unlike tangible inventions such as fire and the wheel, money, possesses an immaterial
nature. It exists as a conceptual construct, lacking inherent value, and its significance is derived solely from the
subjective importance we attribute to it. This intangible nature of money often distinguishes it from other notable
inventions in the collective human consciousness~\cite{smith2010wealth}.

Notwithstanding the illusory nature of money, its significance remains unaffected. Before to the establishment of
monetary systems, human societies engaged in the direct exchange of goods and services, known as the Barter system. In
this system, individuals traded commodities without an assigned intrinsic value, relying solely on subjective
evaluations of desired items. Consequently, each transaction was contingent upon the willingness of the parties involved
to forfeit possessions in pursuit of their desired commodities. Such an exchange mechanism resembled a game-like
scenario~\cite{durlauf2016new}. If I desired vegetables for my meal but my only possession was cattle, I would be
obliged to offer one of my animals in exchange for bags of vegetables. Similarly, if I required footwear but specialized
in tent production, I would have to surrender an entire tent to obtain a pair of slippers. This barter-based system
reveals a prominent issue known as asymmetry. As a tent-maker, the exchange of an entire dwelling for simple footwear
would undoubtedly leave me feeling disadvantaged. The absence of a standardized medium of exchange presented significant
challenges for facilitating agreements between individuals with disparate needs. Moreover, the reliance on a fortuitous
occurrence of complementary wants, wherein two individuals simultaneously sought reciprocal possession, further
complicated matters, rendering the process inefficient~\cite{goodhart1998two}.

Our monetary system serves not only as a medium of exchange but also as a store of value. However, prior to the advent
of money, certain individuals were unable to effectively preserve their wealth, through no fault of their own. Consider
the scenario of a farmer selling tomatoes and a tent maker. The tent maker has the ability to amass a substantial
portfolio of real estate in the form of tents, which can be bartered year-round with individuals in need of shelter.
Consequently, the tent maker has the opportunity to accumulate wealth. In contrast, the farmer selling tomatoes can only
engage in barter transactions during the tomato season. Moreover, due to the perishable nature of tomatoes, long-term
storage is not feasible. Thus, despite exerting comparable efforts in their respective businesses, the farmer had no
viable means to sustain wealth throughout the year~\cite{de2016origins}. There's also the problem of having something
that only a very few people want. Nowadays, when starting a business, you're often told to find a niche. A small group
of people who are very interested in what you have to offer. Before money was a thing, that advice would have left you
with nothing worth bartering.

In societies where possessions in high demand, such as weapons, animal skins, and salt, held significant value,
individuals who possessed such commodities acquired substantial wealth. The awareness that these items were universally
sought-after prompted individuals to engage in anticipatory buying, even if immediate need was absent, to secure future
trading opportunities. As a consequence, the emergence of commodity money ensued, whereby goods and services were
exchanged for commonly recognized items such as salt or weapons, facilitating subsequent transactions with other
parties~\cite{polanyi1965trade}.

Humanity progressed beyond direct barter, encompassing a diverse range of commodities including salt, weapons, and
minutecollectibles like shells and beads. This evolution introduced a more efficient method of trade and exchange.
Rather than directly swapping goods and services, individuals adopted the practice of using arbitrary objects as
intermediary placeholders of value, effectively functioning as IOUs (I Owe You). Subsequently, these placeholders could
be utilized to acquire desired goods and services from others. This concept proved remarkably ingenious, ultimately
leading to a global transition from the Barter system to the monetary exchange system~\cite{graeber2012debt}. However,
there has been a persistent limitation associated with this form of exchange. In order for currency to exhibit intrinsic
value, it requires a degree of scarcity~\cite{smith2010wealth, ricardo1821principles}. The more easily accessible an
item is, the lower its perceived worth~\cite{marshall2009principles}. When an item is readily obtainable by anyone, its
value diminishes considerably. As a result, substances such as sand or shells, which can be effortlessly collected from
any beach, do not effectively function as indicators of value~\cite{principlesmenger, hicks1936keynes}.

In approximately 770 BC, China witnessed the emergence of the earliest metal coins, marking a significant milestone in
the evolution of currency. As a tribute to their historical currency systems, the Chinese craftsmen ingeniously crafted
miniature replicas of tools that were previously utilized as forms of exchange. To ensure convenient handling, the coins
were deliberately designed in a circular shape, allowing easy retrieval from pockets without causing any discomfort to
the fingers. These coins were predominantly cast using bronze, thereby bestowing intrinsic value upon them. This
transition marked a pivotal moment in history, as money transformed from a mere symbol to a tangible entity of worth.
The scarcity of bronze, a resource not readily available on any beach, further amplified the significance of these
coins~\cite{li2003, hartill2005}. During this period, the concept of money had not yet deviated from material reality.
The valuation of a coin corresponded directly to the intrinsic value of the metal constituting the coin. For instance, a
coin crafted from 1 gram of gold possessed an equivalent worth of precisely 1 gram of gold. This quantifiable attribute
allowed for straightforward verification through direct measurement, enabling individuals to visually ascertain that the
coin indeed comprised 1 gram of gold.

The realization of the potential power of money was swift among Kings and Rulers~\cite{cribb1991}. This understanding
led to the creation of the first official money mint by Alyattes, the King of Lydia, around 600 BC. These coins were
minted from a blend of silver and gold, with each coin featuring a distinctive image serving as a denomination.
Consequently, individuals could effortlessly determine the value of their metal possession by observing the pictorial
representation on the coin's surface~\cite{deVries2008}. The pursuit of greater wealth among Kings led to the
devaluation of coins through the reduction of precious metal content and the inclusion of cheaper
metals~\cite{weatherford1997}. This resulted in the divergence between the face value and actual worth of circulating
coins, establishing the illusion of money. The value of coins became divorced from the intrinsic value of their metal
composition, relying instead on the dictates of rulers and financial institutions~\cite{ferguson2009}. As an example,
the British Pound Sterling ceased to represent a fixed quantity of Sterling Silver and instead denoted a unit of
currency determined by authoritative decree.

The emergence of international trade exposed the impracticality of metal coins, leading to the introduction of IOU
certificates by the Kings to facilitate long-distance transactions~\cite{graeber2011}. These certificates, bearing the
King's stamp, gained trust and were believed to hold value, as they were expected to be exchangeable for equivalent
coins. Initially, this belief corresponded to reality. With the proliferation of IOU certificates in circulation, the
necessity for physical coins diminished. Ultimately, the value of the certificates became divorced from their direct
convertibility into gold and silver coins. Instead, their value relied on collective trust and shared
belief~\cite{ingham2004}. This shift allowed the paper certificates to retain value based on our perception, even in the
absence of an immediate exchange for tangible precious metals.
