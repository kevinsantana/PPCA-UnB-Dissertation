\subsection{The Illusion of Money}
From Ancient Kings to modern-day governments and Central Banks, money has remained an illusion. A mere representation of
whose value is determined by the importance people place on it.

The ten thousand Singapore Dollars banknote, while no longer in production, remains the highest denomination in circulation
~\cite{goodhart1998}. Despite its intrinsic production cost of fewer than 20 cents, the value of this paper note is upheld
by the illusion perpetuated by the fiat currency system~\cite{gupta2019}. Presently, its monetary equivalence to seven
thousand three hundred and forty-five US Dollars enables its utilization in acquiring substantial assets such as houses,
cars, and even valuable commodities like gold.

“Fiat” is the fancy word we use to describe the modern-day illusion. It's a Latin word that translates to “let it be done.”
It's a decree by the government that, in the case of money, determines what its value is and enforces it as legal tender
~\cite{reinhart2018, friedman2000}.

The elusive nature of money often evades careful consideration, yet akin to historical rulers, contemporary governments
possess an understanding of the influential power of currency and persistently strive for its accumulation. Recognizing
that the possession of greater quantities of these paper instruments equates to amplified authority, governments adopt the
approach of generating additional currency ex nihilo. For instance, in the scenario where the United States government
necessitates \$340 million dollars to procure an F-22 jet, it possesses the capacity to create the required funds through
the act of monetary printing~\cite{graeber2011, mankiw2014}. But there is one problem with this: \textbf{inflation}.

The fundamental attribute of money lies in its role as a medium of exchange, conferring value upon it~\cite{mankiw2014}.
Consequently, the quantity of money in circulation should align with the aggregate production of goods and services. Should
the issuance of money exceeds the availability of goods and services, with all else remaining constant, the resultant effect
is an escalation in prices and a subsequent devaluation of the currency itself. This concern resonates with economists and
the general population, including individuals such as ourselves,~\cite{blinder2010}, particularly in the context of the
current global reserve currency, the United States Dollar.

The year 2020 proved to be an exceedingly challenging period for the world at large, as the onset of the pandemic necessitated
the temporary closure of numerous economies, resulting in a considerable reduction in the availability of goods and services
and a marked decline in overall economic output, as outlined in the World Economic Outlook report by the International
Monetary Fund~\cite{imf2020}. To avert economic collapse and the potential disintegration of societal systems,
the US government embarked on an unprecedented scale of monetary expansion, surpassing any previous instances of currency
printing in its history~\cite{blinder2020}. As of 2021, the current state of affairs reveals a considerable expansion of
the US dollar supply, with approximately 40\% of the existing currency having been printed within the last 18 months
~\cite{fedmoneysupply}. This substantial increase in the money supply with the country's output has raised concerns
regarding the potential for significant price inflation~\cite{Blanchard2021}. Observable evidence of this trend is already
apparent in the substantial rise in commodity prices, such as the tripling of lumber prices compared to a year ago. Additionally,
discernible price increases can be observed in everyday experiences, including slight increments in prices at favorite restaurants,
such as a modest 20-cent rise in the cost of guacamole at Chipotle~\cite{BLS}. Although the provision of stimulus and unemployment
checks by governments to their citizens may initially appear beneficial, it entails a double-edged sword. While it undoubtedly
assists individuals in dire economic circumstances, it also introduces challenges. Presently, the combined factors of inflationary
pressures and an economic slowdown have created difficulties for individuals seeking suitable employment opportunities, not
solely due to a lack of willingness but also because certain job options may be less desirable than available alternatives
~\cite{cbo2020, kahn2020}.

An illustrative example can be observed in the United States, where the law does not mandate a minimum wage for individuals
working as waiters or waitresses~\cite{dol}. Consequently, some employees in these roles receive meager hourly wages, such
as \$2 to \$3, with tips constituting a substantial portion of their earnings. However, due to the implementation of various
restrictions and regulations nationwide, coupled with a decrease in customer traffic, there has been a reduction in both
customer volume and disposable income, thereby leading to a decline in tip revenue~\cite{bls2022}. Inadequate income for
employees may result in higher turnover rates as financial needs are not being met. This situation poses a significant risk
to businesses, as the lack of a sufficient workforce can ultimately lead to business closure, setting in motion a cascading
effect~\cite{azar2020}. A valid concern arises regarding the motivation to actively seek employment when the potential income
from unemployment and stimulus checks surpasses that from being employed. This circumstance prompts an examination of the
available options. Notably, the Federal Reserve of the United States employs a strategic approach to injecting funds into
the economy, a process that may not be widely acknowledged, thus stimulating economic activity without substantial public
scrutiny~\cite{frb}. Consequently, the relative attractiveness of alternative income sources may influence individual's
decision-making regarding employment prospects~\cite{cbo2021}.

The United States had accumulated a staggering national debt of \$29 trillion before 2020, an astounding and challenging
figure to comprehend \cite{usdebt}. This debt is primarily financed through the issuance of bonds and Treasury notes, which
are essentially contractual instruments offering repayment of a predetermined principal sum alongside interest
~\cite{treasurysecurities}. Presently, investing in a 10-year U.S. Treasury bond would yield a modest return of 1.23\%
upon maturity. Therefore, investing \$1,000 today would result in a nominal return of a mere \$12.30 by 2031. However,
this return fails to keep pace with the targeted inflation rate, projected to be around 2\% annually~\cite{frbinflation}.
It should be noted that actual inflation rates may surpass the target, although that discussion is beyond the scope of the
current context. Consequently, investing in government notes issued by one's own country, whose currency is utilized in
daily transactions, leads to a gradual erosion of purchasing power over a decade. Irrespective of these concerns,
financial institutions, businesses, and individuals worldwide participate in the acquisition of bonds and treasury notes,
thereby providing governments with discretionary funds for utilization~\cite{treasurysecurities}. However, when the government
confronts the need to fulfill its debt obligations, the previously obtained funds have been fully expended. Consequently,
the government initiates repurchases of treasuries and bonds, confining such transactions to prominent financial institutions
and remunerating them through freshly created money, effectively conjured from nothingness. The Federal Reserve, for
instance, has repurchased over \$1 trillion in bonds since March 2020, with plans to persist in such actions well into the
future~\cite{federalreserve}.

Through government injections, banks are empowered to expand their lending activities, thereby increasing interest income
and fostering economic growth~\cite{federalreserve}. However, this surge in lending simultaneously expands the aggregate
money supply, leading to a depreciation in the value of each dollar. The implementation of multi-trillion dollar
stimulus payments and infrastructure packages raises questions regarding the sustainability of such practices. The influx
of new money results in a devaluation of existing money, whereby the balance in an individual's bank account remains unchanged,
yet its purchasing power diminishes owing to the influx of newly minted money~\cite{currencydevaluation}. Consequently,
the retention of wealth in a fiat currency like the US dollar progressively erodes its value, ultimately impeding the ability
to acquire goods and services despite nominal bank balances.

The reality that money is nothing but an illusion is one that we must all embrace. Only then will the path to financial
freedom become clearer. Understanding that money does not have any intrinsic value in itself but instead only inherits the
value we give it.

As the money supply continues to expand, the purchasing power of each dollar held in one's possession inevitably
erodes, whereas the dollar-denominated value of global assets tends to appreciate~\cite{moneyprinting}. Nevertheless, this
perceived growth can be likened to an optical illusion, employing deceptive mechanisms. Despite the seemingly unrelenting
ascent of the stock market, the underlying reality is far from reassuring. The relentless depreciation of the currency
compounds the situation, eroding its value daily. For example, if the Dow Jones Industrial Average, which serves
as a benchmark for the performance of 30 major US companies, were denominated in terms of gold rather than USD, it would
become apparent that its value has essentially stagnated since 1997~\cite{stockmarketillusion}.

But what's the end goal of all of this? With fiat and an unlimited supply of money, will the value of each currency just
continue to decrease until the end of time? Will the gap between the rich and the poor continue to grow wider? Or are we
going to finally fix a problem as old as man itself and stop placing our financial success in the hands of those who are
destroying it day by day? Money is corruptible.

Only time will tell, but just to know, there is a way out: \textbf{Bitcoin}.
