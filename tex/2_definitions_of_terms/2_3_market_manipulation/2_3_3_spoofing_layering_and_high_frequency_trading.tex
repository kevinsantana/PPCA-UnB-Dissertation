\subsection{Spoofing, Layering, and High-Frequency Trading}

Spoofing and layering are not limited to electronic trading but can also occur in manual and non-electronic trading
settings. However, the majority of spoofing cases that have been prosecuted thus far have occurred in the context of
automated (algorithmic) trading, particularly within the realm of high-frequency trading (HFT). It is expected that in
the future, spoofing will predominantly be carried out by traders utilizing algorithms and engaging in HFT practices.
The association between spoofing and HFT is undeniable, as HFT provides a conducive environment for executing spoofing
and layering schemes, which can enhance the financial gains for the traders involved. HFT's speed enables traders to
minimize the risk of other market participants trading against their spoof orders by swiftly canceling those orders in
response to upward price movements. Given that HFT has opened avenues for spoofing and layering, it is essential to
consider spoofing enforcement within the broader framework of HFT regulation.

HFT has experienced substantial growth in the past 15 years, reaching its peak in 2009, and continues to account for a
significant portion of trading volume in both U.S. equities and futures markets. Purported advantages of HFT include
reduced short-term market volatility, narrower bid-ask spreads for large-cap stocks, and increased market liquidity and
efficiency. However, these advantages are accompanied by perceived drawbacks such as declines in market integrity,
fairness, and the quality of liquidity. Concerns regarding these disadvantages have prompted the introduction of federal
legislation aimed at regulating HFT, although no such bill has been enacted to date. Additionally, regulatory bodies
such as the U.S. SEC, the U.S. CFTC, and exchanges have largely refrained from implementing specific regulatory measures
targeting HFT \cite{chilton2012perspectives}.

Proposed regulations aimed at addressing HFT encompass a range of measures, including: (1) the implementation of speed
bumps to introduce delays in orders or information, thereby reducing the speed advantage enjoyed by HFT firms over other
investors; (2) the requirement for proprietary HFT firms meeting specific criteria to register with regulatory bodies
such as the U.S. Commodity Futures Trading Commission (CFTC), U.S. SEC, or the Financial Industry Regulatory Authority
(FINRA); (3) the introduction of a financial transaction tax specifically targeting HFT firms; (4) the prohibition of
co-location, which involves HFT firms and brokers paying exchanges to place their servers in the same physical location
to minimize latency periods; and (5) the implementation of order cancellation fees. Although these proposals have faced
considerable obstacles in the United States, certain exchanges have taken steps to introduce speed bumps. It is worth
noting that HFT is not limited to the United States, and other jurisdictions have been more proactive in regulating such
trading practices. Notably, the European Union's MiFid II Directive (effective in 2018) and Market Abuse Regime
(effective in 2016) represent the world's most comprehensive set of rules governing HFT. These regulations cover various
aspects, including market access for HFTs, algorithm monitoring, redefinition of market manipulation in light of HFT,
classification of spoofing and layering as forms of market manipulation, and the exclusion of intent as an element for
civil offenses related to spoofing and layering. Arguments advocating for harmonizing the U.S. approach to HFT with that
of the EU hold merit. However, it is important to note that while the United States has not expressly criminalized
spoofing, the United Kingdom has not yet seen criminal prosecutions for spoofing \cite{angel2020spoofing,
chilton2012perspectives, cataldo2019making}.