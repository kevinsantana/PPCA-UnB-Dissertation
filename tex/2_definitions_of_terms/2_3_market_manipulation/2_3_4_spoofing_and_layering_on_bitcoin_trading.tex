\subsection{Spoofing and Layering on Bitcoin Trading}
In the context of the Bitcoin market, spoofing remains a concern. Although the legality of spoofing in the cryptocurrency
domain is still evolving, engaging in spoofing practices is generally discouraged and potentially illegal in many jurisdictions.
Traders attempting to manipulate the Bitcoin market through spoofing may face regulatory consequences, similar to the case
of a California day trader who was penalized by the SEC for a spoofing scheme.

The impact of spoofing on the Bitcoin market is a topic of debate. Some argue that spoofing can artificially affect
short-term price movements, causing panic selling or buying among leveraged traders. However, long-term investors,
commonly referred to as hodlers, are less likely to be significantly affected by short-term market manipulations. They
focus on the fundamental factors driving Bitcoin's value, such as increasing adoption and scarcity.

Bitcoin market dynamics differ from traditional financial markets due to its decentralized nature and limited supply.
The presence of spoofing, while acknowledged, is considered less critical in the long-term valuation of Bitcoin. Hodlers,
who accumulate and hold Bitcoin for extended periods, recognizing instances of spoofing can serve as confirmation for the
continuation of price movements in a specific direction.
One might proceed to analyze a real order book example, specifically Bitcoin to the dollar, wherein buying Bitcoin is priced
at \$4,200 and selling it is valued at \$4,190. Notably, there are 77 Bitcoin available for purchase at \$4,200 and 30
Bitcoin was available at \$4,190. The "asks" represent the selling offers, denoted in red, while the "bids" reflect the buying
bids, depicted as green (represented by yellow in the absence of an alternative color). If a significant amount of selling
offers (red) and a limited number of buying bids (green) are observed, it indicates substantial selling pressure, implying
a potential downward price movement.

Conversely, if there is a considerable number of buying bids (green), it suggests a potential upward price movement. This
analysis is based on the assumption that to drive the price up, the existing sell offers must be gradually consumed.
Consequently, if a surplus of sell offers exists (red), it indicates a substantial "ask wall." Conversely, a large number
of buy bids (green) represents a significant "bid wall," implying a potential price increase. Returning to the presentation,
a scenario is presented where 450 coins are available at a price of \$10 and 150 coins are available at \$9. In this case,
assuming the market has been declining, the individual intends to sell their coins at the highest possible price, ideally
\$10, without placing an actual trade.

To manipulate the market without executing a trade, the individual places a deceptive buy wall, known as spoofing. This
is achieved by placing a series of orders at \$8 and \$7, thereby inflating the apparent volume available at these levels.
However, the individual has no intention of buying coins at these levels. By creating a facade of significant buying
interest, the market is deceived into perceiving the end of the downturn, resulting in increased buying activity.
Consequently, the price rises and the individual cancels their deceptive orders. Capitalizing on the price movement, the
individual sell their coins at the inflated price, reducing their losses compared to selling at the initial lower price.

Subsequently, the market may resume its downward trajectory. To detect spoofing, attention must be paid to the order book,
where a sudden influx of orders at a particular price level, accompanied by a low number of orders, can indicate either
genuine large-scale trading activity or the presence of spoofing. Similarly, an abnormally high volume at a particular
price level may indicate either substantial buying interest or spoofing.

This analysis should be contextualized within the price movement, considering factors such as volume trends and market
conditions. Traders can potentially profit from spoofing by observing the initial price movement triggered by the spoofer,
anticipating the subsequent continuation of the price trend, and executing appropriate trading strategies, such as
short-selling or buying.
