\section{Cryptocurrencies}
In this section, we present viewers with the definition of cryptocurrency, how does a cryptocurrency transaction work
and what is an anomaly on this context and the types of anomalies that can raise from cryptocurrency transactions. A
more in-depth explanation of Bitcoin and its protocol can be found on appendix a --- Bitcoin: A Peer-to-Peer Electronic
Cash System.

Digital tokens known as cryptocurrencies enable direct online payments between individuals. Unlike national currencies,
which derive value from legal tender status, cryptocurrencies lack legislated or intrinsic value and are essentially
valued based on market demand. Bitcoin and Ether are among the most recognized cryptocurrencies, with several others
existing in the market.

\subsection{How Does a Cryptocurrency Transaction Work?}
Cryptocurrency transactions involve the transmission of electronic messages across the entire network, containing
instructions detailing the parties' electronic addresses, the currency quantity for trade, and a time stamp.

If Alice intends to transfer one unit of cryptocurrency to Bob, she initiates the transaction by broadcasting her
instructions through an electronic message visible to all network users. This transaction joins a queue of recent
transactions awaiting compilation into a block, which is essentially a group of the most recent transactions. Since the
system is not instantaneous, the transaction patiently resides with other recent transactions until they are assembled
into a block.

The information within the block is encrypted into a cryptographic code, prompting miners to engage in a competitive
process to decipher the code and append the new block of transactions to the blockchain. Upon a miner successfully
solving the code, other network users verify the solution, reaching a consensus on its validity. Subsequently, the new
block is seamlessly added to the end of the blockchain, confirming Alice's transaction. This confirmation, however, is
not immediate, requiring the processing of six blocks of transactions to ensure users' certainty about the success of
their transactions.

\subsection{Anomalies and Cryptocurrency Transactions}

