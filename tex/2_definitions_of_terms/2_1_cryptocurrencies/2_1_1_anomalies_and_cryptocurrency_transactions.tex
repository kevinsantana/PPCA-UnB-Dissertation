\subsection{Anomalies and Cryptocurrency Transactions}
Anomaly detection is the scientific pursuit of identifying patterns within data sets that deviate from anticipated
behaviors. Such deviant patterns are commonly denoted as anomalies, outliers, discordant observations, exceptions,
aberrations, surprises, peculiarities, or contaminants across diverse application domains. Within the domain of anomaly
detection, the terms `anomalies' and `outliers' are frequently employed, at times interchangeably. This scientific
discipline finds broad applicability in various domains, including but not limited to fraud detection in credit cards,
insurance, or healthcare, intrusion detection in cyber-security, fault detection in safety-critical systems, and
military surveillance for the identification of enemy activities \cite{10.1145/1541880.1541882}.

Anomalies can occur at any level of abstraction, from individual data points to entire systems, and can be caused by a
wide range of factors, including human error, software bugs, hardware failures, and malicious activity. As such, on this
research we expect to encounter anomalies on the form of fraudulent activities, on purpose or not.

\begin{definition}
    \textit{An anomaly is an observation or a sequence of observations which deviates remarkably from the general distribution of data. The set of the anomalies form a very small part of the dataset.}
\end{definition}

For consistency, we will use the term anomaly throughout this research.

The identification of anomalies in cryptocurrency transactions poses a challenging task, particularly when such
anomalies arise from malicious activities. Malicious adversaries exhibit adaptability by manipulating anomalous
observations to mimic normal patterns, thereby introducing complexity to the definition of normal behavior. Another
obstacle in the realm of anomaly detection in cryptocurrency transactions is the variability in the conceptualization of
anomalies across different application domains. For instance, within the medical domain, a slight deviation from the
norm, such as fluctuations in body temperature, may constitute an anomaly. In contrast, analogous deviations in the
cryptocurrency market domain, such as fluctuations in the value of a specific cryptocurrency, might be considered
normal. Consequently, the direct application of techniques developed in one domain to another is not a straightforward
endeavor \cite{10.1145/1541880.1541882}.

