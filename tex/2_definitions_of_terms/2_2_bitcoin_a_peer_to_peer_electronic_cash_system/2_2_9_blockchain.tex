\subsection{Blockchain}
A distributed ledger system is composed of multiple nodes that broadcast transactions. To ensure consensus on the correct ledger, it is necessary to develop a mechanism that allows all nodes to agree on the validity of each transaction~\cite{el2018review}.

The core concept of the original Bitcoin paper~\cite{nakamoto2008bitcoin} is based on the assumption that a distributed ledger will be trusted if it has been subject to a large amount of computational effort. This idea is implemented through the use of the many-zeroes game, which involves proving that a particular block in the chain contains a hash that is difficult to reverse-engineer.

Rather than hashing the entire ledger repeatedly, it is more efficient to allow for the accumulation of computational effort over time. Transactions are grouped into blocks and added to the chain in a linear fashion, with each new block containing a reference to the previous one. This approach allows for the creation of a tamper-evident history of transactions that is trusted by network participants due to the large amount of computational work required to manipulate it.

%% figure 1.11
\figuraBib{img/chapter-2/2_2_9_blockchain/11_blocks.png}{Blocks on a blockchain}{3blue1brown_blocks}{blocks}{width=.85\textwidth}%

The block is a collection of transactions enclosed with a unique identifier, known as proof-of-work (PoW), which serves as evidence of the computational effort expended in validating the block. In PoW schemes, the miner must solve a complex mathematical problem to validate the block and add it to the blockchain. The difficulty level of this problem is determined by the target number of leading zeros required in the hash value of the block.

%% figure 1.12
\figuraBib{img/chapter-2/2_2_9_blockchain/12_block-ordering.png}{Because blocks are chained together like this, instead of calling it a ledger, this is commonly called a “blockchain”}{3blue1brown_blockchain}{blockchain}{width=.85\textwidth}%

A block is considered valid if it contains a proof-of-work (PoW) value, analogous to how a transaction is only considered valid when signed by its sender. Additionally, maintaining the integrity of the blockchain requires that blocks are not rearranged as this would disrupt the transaction history. To address this issue, each new block must begin with the hash of the previous block (hash-based chain), ensuring that the order of the blocks remains consistent.

\subsubsection{Block Creators: Miners}
To maintain the integrity of our ledger after it has been split into blocks, we have introduced a new process for adding
new transactions. This involves grouping together transactions into blocks and computing a proof of work. As part of our
updated protocol, anyone in the world is allowed to act as a "block creator". The responsibility of the block creator is
to listen for broadcasted transactions, collect them into a block, and then perform a significant amount of computational
work to find a special number that will result in the hash of the block starting with 60 zeros. This computed hash value
is then broadcasted to the network as proof of work~\cite{wood2014ethereum}.

A special transaction can be included at the beginning of each block, where the creator is rewarded with a predetermined
amount of digital currency. This practice has been suggested as a means of compensating individuals for their efforts in
constructing blocks within a distributed ledger system~\cite{ding2020incentive}.

%% figure 1.13
\figuraBib{img/chapter-2/2_2_10_block_creators_miners/13_block-reward.png}{Block reward}{3blue1brown_block_reward}{reward}
{width=.45\textwidth}%

The block reward is a unique exception to our usual transaction acceptance rules in the Ledger Dollar economy, as it does
not require signature verification and increases the total number of currency units with each new block.

The process of creating blocks, known as
"mining", involves a significant amount of work and introduces new currency into the economy. However, when discussing
miners, it is essential to understand that they are primarily focused on listening to transactions, constructing blocks,
broadcasting them, and receiving newly minted currency as a reward for their efforts.

For miners, each block can be thought of as a miniature lottery where individuals guess numbers rapidly until one person
finds a combination that results in a hash starting with many zeros, earning the resulting reward. In contrast to mining,
non-mining Bitcoin users no longer need to record all individual transactions on their personal ledger. Instead, they can
simply monitor block production and rely on the fact that these blocks contain verified transactions. This approach is more
manageable than maintaining a comprehensive transaction ledger.

In the consensus algorithm used by Bitcoin and other cryptocurrencies, a mechanism known as the "longest chain rule" is
employed to resolve potential conflicts between competing blocks. Specifically, if two miners broadcast distinct blockchains
with conflicting transaction histories, the system defers to the one that has been the longest in terms of cumulative
proof-of-work effort expended on it, which is assumed to be more resistant to manipulation~\cite{buterin2014next}. If
there is a tie between two competing blocks, it may be necessary to wait for additional information to determine which
block is longer. This process relies on the assumption that the longest chain represents the most widely accepted version
of the blockchain. However, this approach has been subject to criticism due to its reliance on proof-of-work mechanisms,
which require significant computational effort and can lead to centralization.

\input{tex/2_definitions_of_terms/2_2_bitcoin_a_peer_to_peer_electronic_cash_system/2_2_11_attempt_fraud_on_the_blockchain.tex}
\input{tex/2_definitions_of_terms/2_2_bitcoin_a_peer_to_peer_electronic_cash_system/2_2_12_ledger_dollars_vs_bitcoin}
