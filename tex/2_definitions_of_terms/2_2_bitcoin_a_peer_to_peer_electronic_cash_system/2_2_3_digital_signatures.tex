\subsection{Digital Signatures}
Digital signatures provide a means to ensure the authenticity and integrity of electronic transactions. The use of digital
signatures allow recipients to verify that the information sent by a sender is what they intended to send, thereby establishing
trust in the transaction~\cite{stinson2018cryptography}.

The concept described here is similar to a handwritten signature, whereby Alice can add a message or proof of approval to
a transaction that cannot be easily replicated by others. This is achieved through the use of digital signatures, which
are based on cryptographic algorithms and provide a secure method for verifying the authenticity of a message or
transaction~\cite{elgamal1985public}. The infeasibility of forging a signature is ensured through the use of advanced
encryption techniques that make it difficult for unauthorized parties to tamper with or counterfeit digital
signatures~\cite{elgamal1985public}.

%% figure 1.2
\figuraBib{img/chapter-2/2_2_3_digital_signatures/2_digital_signature.png}{Digital signature}{docusign_digital_signature}
{signature}{width=.85\textwidth}%

It may seem counterintuitive at first, but digital signatures can be implemented in a way that prevents forgery. In this
context, a digital signature is a function of two elements: the private key, which only the signatory possesses, and the
message being signed~\cite{diffie2022new}. This means that even if an attacker were able to copy the initial signature,
subsequent attempts to use it would result in a different value due to the unique relationship between the private key
and the message.

In cryptography, a signature function is only effective if there exists a verification function to confirm its validity
\cite{stallings2006cryptography}. The mechanism for this involves generating a public-private key pair consisting of two
strings of 1's and 0's. The private key, also known as the \textit{secretkey}, is often abbreviated as $sk$ while the public
key is denoted as $pk$. As suggested by their names, the secret key should be kept confidential~\cite{dss}.

A digital signature scheme can be defined as a set of two operations: one for generating a digital signature on a given
message, denoted as $\text{Sign}$, and the other for verifying the authenticity of a purported signature, denoted as
$\text{Verify}$. These functions are typically implemented as follows:

\begin{enumerate}
    \item \emph{Signing function} $\text{Sign}$: This operation takes as input a message $m \in \{0, 1\}^*$, and produces
    a digital signature $\text{Sign} \in \mathbb{Z}_q^*$, where $q$ is a prime number. The security of the scheme is typically
    guaranteed by the assumption that it is computationally infeasible to compute the discrete logarithm in the underlying
    finite field, $\mathbb{Z}_q$.

    \item \emph{Verification function} $\text{Verify}$: This operation takes as input a message $m \in \{0, 1\}^*$, a digital
    signature $\text{Sign} \in \mathbb{Z}_q^*$, and the public key $(pk, sk)$, where $pk = g^x$ for some generator polynomial
    $g \in \mathbb{Z}[X]$ of degree $n-1$ and $x \in \mathbb{Z}_q$. The verification function outputs a Boolean value indicating
    whether or not the given signature is valid, i.e., $\text{Sign}(m) = g^y \mod q$, where $y \in \mathbb{Z}$ is the unique
    integer such that $g^{y\bmod{n}} \equiv sk \pmod{q}$.

\end{enumerate}

The signing process requires employing the private key. The objective is that if Alice alone possesses her private key,
then she is the only individual capable of generating a digital signature. If this key is compromised the security of the
system is significantly undermined. The $\text{Verify}$ function serves as a means of determining whether a given message
bears a valid digital signature generated using the corresponding public key. It should return True when applied to an
authentic signature and False for all other signatures.

The security of a digital signature scheme relies on the secrecy of the private key used to generate the signature. However,
it is theoretically possible for an attacker to brute-force the public key and find a valid signature by exhaustively trying
different potential signatures until one returns true~\cite{boneh2001short}. In the case of Bitcoin's digital signature scheme,
there are $2^{256}$ possible signatures due to the large number of bits in the hash function used for signature generation
~\cite{dss}. However, this number is so large that it makes brute-force attacks on the public key infeasible, providing
a high level of security for Bitcoin's digital signatures.
