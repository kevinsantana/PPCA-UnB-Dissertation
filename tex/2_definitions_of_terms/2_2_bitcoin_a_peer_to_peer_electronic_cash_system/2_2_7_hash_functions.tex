\subsection{Hash Functions}
Cryptographic hash functions are the primary tool utilized by Nakamoto's solution to this puzzle. These functions take in arbitrary messages or files as input and produce a fixed-length string of bits referred to as the "hash" or "digest" of the message, which is intended to exhibit randomness. The output of this process is deterministic and consistent for a given input, but minor alterations to the input can lead to drastically different hash values.

The property of unpredictability in the output changes when slightly changing the input is what makes SHA256 a cryptographic hash function~\cite{dang2015secure}. This means that it is computationally infeasible to compute the original message from its hash value in reverse direction~\cite{butin2017hash}. Therefore, given a specific hash value such as $1001111100111100\ldots$, there is no efficient method to determine the corresponding input message other than brute-force guessing and checking with random inputs.

Given the provided function, what is the empirical evidence indicating a significant correlation between a specified set of Bitcoin transactions and an exceptional computational expenditure? \emph{Proof-Of-Work}.
