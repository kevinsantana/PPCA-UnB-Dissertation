\subsection{Proof-Of-Work}
The task described involves manipulating a collection of transactions (enclosed within a container), whose hash value is computed utilizing the SHA256 algorithm. The objective is to modify a specified element within the container, such that the resulting hash commences with at least six consecutive zeros.

Achieving a solution to this problem is indeed possible, albeit requiring a considerable amount of time. Due to the inherent unpredictability of the hash function's output, the prevailing method for tackling this challenge remains a process of trial and error~\cite{Dworkin2001}.

As the number of required leading zeroes increases, the difficulty of the problem escalates exponentially. Consider a scenario where an individual presents you with a list of transactions and asserts that they have identified a special number. They claim that by appending this number to the end of the transaction list and applying the SHA256 hash function to the entire sequence, the resulting output will exhibit 30 leading zero bits.

Assessing the level of difficulty involved in discovering the aforementioned number necessitates a thoughtful analysis. It is evident that the task likely posed significant challenges. When considering a randomly selected message, the probability of the resulting hash beginning with 30 consecutive zeroes is 1 in $2^{30}$, which corresponds to approximately 1 in a billion~\cite{Dworkin2001}. Consequently, it is highly probable that the individual in question had to iterate through approximately one billion distinct guesses before successfully identifying this specific value.

%% figure 1.10
\figuraBib{img/chapter-2/2_2_8_proof_of_work/10_30-zeroes.png}{There is no better way than guess and check for the special hash}{3blue1brown_guess_and_check}{guess}{width=.85\textwidth}%

Nevertheless, what proves intriguing is that once the number is known, its verification as a hash commencing with 30 zeros can be efficiently conducted. This verification process offers the ability to ascertain the substantial effort expended by the individual without necessitating the replication of the original labor. Termed as \emph{proof-of-work}, this number holds significance.

It is crucial to emphasize that the entirety of this endeavor is intrinsically linked to the underlying list of transactions. Even a slight modification to any transaction would result in a completely altered hash, compelling a full repetition of the laborious process to identify a new number that yields a hash with 30 zeros~\cite{nakamoto2008bitcoin}.
