\subsection{Ledger}
In blockchain systems, transactions are recorded on a distributed ledger and secured through cryptographic techniques.
Specifically, each transaction needs to be signed by its corresponding private key, which ensures its authenticity and
non-repudiation~\cite{stinson2018cryptography}. The signature generated for a given transaction is unique and dependent
on the content of that transaction, making it impossible to reuse signatures from one transaction to another~\cite{bruce1996applied}.
However, there is an issue with this approach. Suppose Alice signs a transaction, such as "Alice pays Bob \$100", which
is then recorded on the blockchain. Although Bob cannot forge Alice's signature on new messages, he could still copy that
same line multiple times and submit it to the network. Since the message/signature combination is still valid, these duplicate
transactions may be accepted by the network and included in its consensus state~\cite{swan2015blockchain}.

%% figure 1.3
\figuraBib{img/chapter-2/2_2_4_adding_lines_to_the_ledger/3_duplicate-transaction.png}{Anyone can create copies of previous
transactions}{3blue1brown_duplicate_transaction}{duplicate-transaction}{width=.85\textwidth}%

The development of digital signatures can address the issue of trust in the initial protocol by introducing unique identifiers
for transactions and requiring a distinct signature for each transaction. This approach has been proposed and implemented
in various cryptographic systems, such as the RSA signature scheme~\cite{rivest1978method} and the elliptic curve digital
signature algorithm (ECDSA). The use of digital signatures not only enhances security but also enables efficient verification
of the authenticity and integrity of electronic messages.

\subsubsection{Removing Cash}
The effectiveness of this system relies on an implicit agreement between individuals to uphold their financial
obligations. Specifically, participants are expected to pay in cash at the end of each month, despite the absence of a
formal enforcement mechanism. However, there is no guarantee that all parties will comply with this arrangement, as
demonstrated by instances where one individual (e.g., Charlie) may accumulate significant debt and subsequently fail to
fulfill their financial obligations.

In this cashless economic system, it may be necessary to revert to cash to settle up if certain individuals owe a
significant amount of money (e.g., Charlie). However, as long as no one falls into debt and the ledger is properly
maintained, the use of cash can be avoided. The ledger alone can function effectively as long as there is a mechanism in
place to prevent excessive spending.

One strategy for managing a cashless economy without resorting to cash settlements is to have all participants deposit
an equal amount (e.g., \$100) into the pot, and record the initial distribution of funds on the ledger. For example,
Alice would receive \$100 in the first transaction, while Bob would receive \$100 in the second transaction, and so on.
By using this approach, individuals can maintain their financial balance without the need for cash transactions.

%% figure 1.4 \figuraBib{img/chapter-1/1_2_5_removing_cash/4_pay-in-100.png}{Cashless economic
% system}{3blue1brown_cashless} {cashless}{width=.85\textwidth}%

Now that we are under a cashless economic system, it is important to prevent double-spending attacks where a user
attempts to spend the same cryptocurrency more than once. One way to accomplish this is by verifying that transactions
are valid before they are added to the ledger. Specifically, if all users on the network start with zero balance (\$0)
and the first two transactions are of \$100 value (Charlie pays Alice \$50 and Charlie pays Bob \$50), then a third
transaction where Charlie pays You \$20 would be invalid. This is because it violates the rule that a user cannot spend
more than they have in their account.

%% figure 1.5
\figuraBib{img/chapter-2/2_2_5_removing_cash/5_invalid.png}{In this new system, we don't allow people to spend more than
they have.}{3blue1brown_invalid}{invalid}{width=.85\textwidth}%

It can be noted that the requirement to ascertain the legitimacy of a transaction necessitates knowledge of the entire
transaction history. This principle applies not only to traditional financial systems but also to decentralized digital
currencies, although opportunities for improvement is present.

%% figure 1.6
\figuraBib{img/chapter-2/2_2_5_removing_cash/6_overdrawn.png} {Now verifying a transaction requires checking the entire
ledger history to make sure nobody overdraws.} {3blue1brown_overdrawn}{overdrawn}{width=.85\textwidth}%

The use of the above ledger system appears to dissociate it from physical cash transactions. If everyone in the world
were to utilize this ledger, one could theoretically conduct all financial transactions solely through the ledger
without any need for conversion to United States Dollars (USD). Many individuals currently perform digital transactions
exclusively while occasionally using physical cash. The latter scenario involves a more intricate system of banks
wherein the balance on a digital account can be converted into USD. However, if one and their associates were to
completely detach their ledger from USD, there would be no guarantee that having a positive balance in the ledger could
translate into physical currency in hand. To accentuate this point, one can stop using the \$ sign, and digital
quantities, on the ledger can be referred to as "Ledger Dollars" (LD).

Individuals possessing Ledger Dollars have the liberty to convert them into US dollars at their discretion. An example
involves Alice offers Bob a zero-value US dollar bill in exchange for him adding and signing a transaction entry to the
shared ledger, wherein Bob pays Alice ten units of Ledger Dollar value. However, the protocol does not explicitly
guarantee the occurrence of such exchanges. Instead, it operates more similarly to foreign currency exchange in an open
market where 10LD is its own independent entity. Additionally, if there is high demand for inclusion within the ledger,
a transaction of 10LD may require a non-zero amount of physical cash. Conversely, if there is a low demand for
participation, it may require only a minimal amount of physical cash.

% %% figure 1.7 \figuraBib{img/chapter-1/1_2_5_removing_cash/7_exchange-currency.png}{Exchange
% currencies}{3blue1brown_exchange_currency} {exchange}{width=.75\textwidth}%

Our ledger has been transformed into a form of currency that operates within a closed system, allowing for peer-to-peer
transactions between individuals without the backing of a state or taxation imposed in the form of Ledger Dollars. It is
important to recognize that, at its core, cryptocurrency can be viewed as a ledger that records the history of financial
transactions, serving as the currency itself. The concept of possessing Bitcoin is simply represented by a positive
balance on the Bitcoin ledger, which is associated with a secret key. This differs from traditional currency systems
where money enters the ledger through cash transactions. In the case of Bitcoin, the process for introducing new money
into the ledger will be discussed in more detail shortly. However, it is important to note that there are fundamental
differences between Ledger Dollars and true cryptocurrencies.

% %% figure 1.8 \figuraBib{img/chapter-1/1_2_5_removing_cash/8_ledger-is-currency.png}{Ledger is
% currency.}{3blue1brown_ledger_is_currency} {ledgerr}{width=.75\textwidth}%

\subsubsection{Distributing The Ledger}
The distributed nature of the blockchain technology used by the ledger system necessitates the use of a centralized platform
for public access and modification of the ledger's contents. However, this raises concerns regarding the trustworthiness
of the entity responsible for hosting the website and regulating the rules governing the addition of new entries to the
ledger. In particular, it is important to identify and evaluate the credibility of the entity that controls the website
and establishes the protocols for updating the ledger.

To eliminate trust in a centralized system where one ledger is maintained, we will replace this with a decentralized approach,
where each individual will maintain their own copy of the ledger. This will enable transactions, such as "Alice pays Bob
100 LD" to be broadcasted and recorded on personal ledgers by all parties involved in the network.

The distributed ledger technology employed by Bitcoin involves the broadcast of transactions by users, which are then recorded
on a decentralized set of records. This eliminates the need for trust in a central authority. However, this system is problematic
due to the possibility of disagreement among participants regarding the correct ledger. For example, when Bob receives a
transaction "Alice pays Bob 10 LD", how can he be certain that everyone else has received and believes in the same transaction?
If even one person does not know about this transaction, they may not allow Bob to spend those 10 Ledger Dollars later.

%% figure 1.9
\figuraBib{img/chapter-2/2_2_6_distributing_the_ledger/9_are-these-the-same.png}{If everyone keeps a unique copy of the
ledger, how can we ensure that everybody agrees on what it should say?}{3blue1brown_ledgers}{ledgerr}{width=.85\textwidth}%

The verification of the integrity and consensus of a blockchain network relies on a distributed ledger system where all
participants maintain a copy of the same transaction history. The trustworthiness of this system is predicated on the assumption
that all nodes will accurately record and remember past transactions, which may be subject to potential inconsistencies
or discrepancies in the event of faulty or malicious behavior. Therefore, it is essential to establish a mechanism for
ensuring that the distributed ledger remains consistent across all participating nodes. The solution proposed by Satoshi
Nakamoto in 2008 for decentralized systems was a method to validate the validity of a growing document, such as a ledger,
without relying on a central authority. This problem was solved through the use of computational work to determine trustworthiness,
where the ledger with the most computational effort invested in it is considered legitimate. The idea is that if an individual
attempts to manipulate the ledger, it would require an impractical amount of computational power, making fraudulent transactions
computationally infeasible. This concept forms the core of Bitcoin and other cryptocurrencies.
