\subsection{Ledger}
In blockchain systems, transactions are recorded on a distributed ledger and secured through cryptographic techniques. Specifically, each transaction needs to be signed by its corresponding private key, which ensures its authenticity and non-repudiation~\cite{stinson2018cryptography}. The signature generated for a given transaction is unique and dependent on the content of that transaction, making it impossible to reuse signatures from one transaction to another~\cite{bruce1996applied}. However, there is an issue with this approach. Suppose Alice signs a transaction, such as "Alice pays Bob \$100", which is then recorded on the blockchain. Although Bob cannot forge Alice's signature on new messages, he could still copy that same line multiple times and submit it to the network. Since the message/signature combination is still valid, these duplicate transactions may be accepted by the network and included in its consensus state~\cite{swan2015blockchain}.

%% figure 1.3
\figuraBib{img/chapter-2/2_2_4_adding_lines_to_the_ledger/3_duplicate-transaction.png}{Anyone can create copies of previous transactions}{3blue1brown_duplicate_transaction}{duplicate-transaction}{width=.85\textwidth}%

The development of digital signatures can address the issue of trust in the initial protocol by introducing unique identifiers for transactions and requiring a distinct signature for each transaction. This approach has been proposed and implemented in various cryptographic systems, such as the RSA signature scheme~\cite{rivest1978method} and the elliptic curve digital signature algorithm (ECDSA). The use of digital signatures not only enhances security but also enables efficient verification of the authenticity and integrity of electronic messages.

\input{tex/2_definitions_of_terms/2_2_bitcoin_a_peer_to_peer_electronic_cash_system/2_2_5_removing_cash.tex}
\input{tex/2_definitions_of_terms/2_2_bitcoin_a_peer_to_peer_electronic_cash_system/2_2_6_distributing_the_ledger.tex}