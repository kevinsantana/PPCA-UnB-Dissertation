\emph{The banking system, as we know it, operates through a fractional reserve lending process, in which only a small
percentage of deposits are held in custody while the rest are lent out to other people or companies. Trust in the
currency transacted and in the government that operates it represents more value than the money itself. It was from this
trust that the 2008 financial crisis took shape: the credit offered by financial institutions and the US government
itself led to speculation in the real estate sector that failed to materialize. This event was the trigger for an idea
previously restricted to enthusiasts of cryptography, economics, mathematics, and computer networks to take the form of
a digital currency, decentralized and far from the power of the state and its financial crises—Bitcoin. Although Bitcoin
is decentralized and does not follow rules imposed by governments, it also has flaws and is the target of more
sophisticated fraud, such as spoofing and layering. The aim of this research project is to explore new ways to detect
fraudulent transactions in cryptocurrency trading, exploiting new machine learning techniques for anomaly detection. We
hope that the results of this research can be transferred to companies responsible for trading crypto assets, as well as
other sectors of society that could benefit from advanced anomaly detection methods.}
