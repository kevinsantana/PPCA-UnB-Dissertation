\section{How EFC Works}
EFC was initially developed for Network Intrusion Detection Systems (NIDS) to
recognize both known and unknown attacks. The classifier is built upon statistical
physics principles, particularly the Potts model, which estimates probability
distributions of normal and anomalous network flows. The primary innovation of
EFC is its ability to generalize well while maintaining computational efficiency,
making it an attractive option for AML applications where real-time detection of
suspicious transactions is crucial \cite{souza2022}.

EFC operates by constructing a statistical model that captures the normal behavior
of transactional flows. It achieves this through the following core principles:

\begin{itemize}
\item \textbf{Inverse Potts Model}: Inspired by statistical physics, EFC models
    interactions between different transaction attributes as energy couplings.
\item \textbf{Hamiltonian Function}: The classifier assigns an energy score to
    each transaction based on its deviation from expected behavior.
\item \textbf{Anomaly Detection}: Transactions with high energy values are classified
    as suspicious, as they deviate significantly from known benign patterns.
\end{itemize}

Mathematically, EFC defines the energy of a transaction flow as:

\begin{equation}
    H(a_1, ..., a_N) = - \sum_{i,j | i<j} e_{ij}(a_i, a_j) - \sum_i h_i(a_i)
\end{equation}

where:
\begin{itemize}
    \item $ a_i $ represents a feature of the transaction,
    \item $ e_{ij} $ is the coupling strength between features $ i $ and $ j $,
    \item $ h_i(a_i) $ is the local field associated with feature $ a_i $.
\end{itemize}

The statistical model underlying EFC is inferred using the maximum entropy principle:

\begin{equation}
    \max_P \left(- \sum_k P(a_1^k, ..., a_N^k) \log P(a_1^k, ..., a_N^k) \right)
\end{equation}

subject to the empirical frequency constraints:

\begin{equation}
    \sum_k P(a_i^k) = f_i(a_i), \quad \sum_k P(a_i^k, a_j^k) = f_{ij}(a_i, a_j)
\end{equation}

for all features $ i, j $, ensuring that the model aligns with the observed distribution.
