\section{Model Inference}
The process of training an EFC involves learning the statistical distribution of
benign transaction flows. This is done by:

\begin{enumerate}
    \item \textbf{Extracting Features}: Transaction data is represented using
    attributes such as sender-receiver patterns, transaction amounts, and frequency.
    \item \textbf{Computing Couplings and Local Fields}: The dependencies between
    different attributes are quantified using a covariance matrix:
    \begin{equation}
        C_{ij}(a_i, a_j) = f_{ij}(a_i, a_j) - f_i(a_i) f_j(a_j)
    \end{equation}
    The couplings are then inferred as:
    \begin{equation}
        e_{ij}(a_i, a_j) = - (C^{-1}){ij}(a_i, a_j)
    \end{equation}
    \item \textbf{Energy Computation}: The energy of a new transaction is calculated
        using the inferred statistical model:
    \begin{equation}
        H(a_1, ..., a_N) = - \sum{i,j | i<j} e_{ij}(a_i, a_j) - \sum_i h_i(a_i)
    \end{equation}
    \item \textbf{Thresholding for Anomaly Detection}: A threshold $ T $ is defined
        based on statistical properties of known benign transactions. Transactions
        with $ H > T $ are flagged as suspicious.
    \item \textbf{Model Refinement}: The model is periodically updated by incorporating
        new benign transaction data, ensuring adaptability to evolving transaction
        patterns.
\end{enumerate}
