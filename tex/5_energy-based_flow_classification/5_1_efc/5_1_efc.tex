\section{Energy-based Flow Classification}
Money laundering poses a significant challenge to financial institutions, particularly
in the cryptocurrency sector, where the pseudonymous nature of transactions provides
opportunities for illicit financial activities. Traditional anti-money laundering
(AML) systems rely on rule-based detection methods, which often fail to identify
novel or evolving fraudulent schemes. To address this issue, machine learning-based
classifiers have gained prominence in anomaly detection within financial transactions.
Among these, the Energy-based Flow Classifier (EFC) presents a novel approach to
detecting both known and unknown fraudulent activities with high accuracy and efficiency.

The Energy Flow Classifier (EFC) is a probabilistic machine learning framework
inspired by statistical physics, particularly the Potts model, which historically
dealt with atomic spins in a lattice. EFC leverages the concept of "flow energy"
to classify network traffic or other data flows \cite{pontes2021}.

This chapter explores the applicability of EFC in AML on the context of cryptocurrency
transactions. It introduces the fundamental principles of EFC, its advantages over
traditional methods, and its potential integration into existing AML frameworks.
