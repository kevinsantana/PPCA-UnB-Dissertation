% chapter 5 - efc

Money laundering poses a significant challenge to financial institutions, particularly
in the cryptocurrency sector, where the pseudonymous nature of transactions provides
opportunities for illicit financial activities. Traditional anti-money laundering
(AML) systems rely on rule-based detection methods, which often fail to identify
novel or evolving fraudulent schemes. To address this issue, machine learning-based
classifiers have gained prominence in anomaly detection within financial transactions.
Among these, the Energy-based Flow Classifier (EFC) presents a novel approach to
detecting both known and unknown fraudulent activities with high accuracy and efficiency.

The Energy Flow Classifier (EFC) is a probabilistic machine learning framework
inspired by statistical physics, particularly the Potts model, which historically
dealt with atomic spins in a lattice. EFC leverages the concept of "flow energy"
to classify network traffic or other data flows \cite{pontes2021}.

This chapter explores the applicability of EFC in AML on the context of cryptocurrency
transactions. It introduces the fundamental principles of EFC, its advantages over
traditional methods, and its potential integration into existing AML frameworks.

% section 5.1 - how efc works
\section{How EFC Works}
EFC was initially developed for Network Intrusion Detection Systems (NIDS) to
recognize both known and unknown attacks. The classifier is built upon statistical
physics principles, particularly the Potts model, which estimates probability
distributions of normal and anomalous network flows. The primary innovation of
EFC is its ability to generalize well while maintaining computational efficiency,
making it an attractive option for AML applications where real-time detection of
suspicious transactions is crucial \cite{souza2022}.

EFC operates by constructing a statistical model that captures the normal behavior
of transactional flows. It achieves this through the following core principles:

\begin{itemize}
\item \textbf{Inverse Potts Model}: Inspired by statistical physics, EFC models
    interactions between different transaction attributes as energy couplings.
\item \textbf{Hamiltonian Function}: The classifier assigns an energy score to
    each transaction based on its deviation from expected behavior.
\item \textbf{Anomaly Detection}: Transactions with high energy values are classified
    as suspicious, as they deviate significantly from known benign patterns.
\end{itemize}

Mathematically, EFC defines the energy of a transaction flow as:

\begin{equation}
    H(a_1, ..., a_N) = - \sum_{i,j | i<j} e_{ij}(a_i, a_j) - \sum_i h_i(a_i)
\end{equation}

where:
\begin{itemize}
    \item $ a_i $ represents a feature of the transaction,
    \item $ e_{ij} $ is the coupling strength between features $ i $ and $ j $,
    \item $ h_i(a_i) $ is the local field associated with feature $ a_i $.
\end{itemize}

The statistical model underlying EFC is inferred using the maximum entropy principle:

\begin{equation}
    \max_P \left(- \sum_k P(a_1^k, ..., a_N^k) \log P(a_1^k, ..., a_N^k) \right)
\end{equation}

subject to the empirical frequency constraints:

\begin{equation}
    \sum_k P(a_i^k) = f_i(a_i), \quad \sum_k P(a_i^k, a_j^k) = f_{ij}(a_i, a_j)
\end{equation}

for all features $ i, j $, ensuring that the model aligns with the observed distribution.


% section 5.2 - model inference
\section{Model Inference}
The process of training an EFC involves learning the statistical distribution of
benign transaction flows. This is done by:

\begin{enumerate}
    \item \textbf{Extracting Features}: Transaction data is represented using
    attributes such as sender-receiver patterns, transaction amounts, and frequency.
    \item \textbf{Computing Couplings and Local Fields}: The dependencies between
    different attributes are quantified using a covariance matrix:
    \begin{equation}
        C_{ij}(a_i, a_j) = f_{ij}(a_i, a_j) - f_i(a_i) f_j(a_j)
    \end{equation}
    The couplings are then inferred as:
    \begin{equation}
        e_{ij}(a_i, a_j) = - (C^{-1}){ij}(a_i, a_j)
    \end{equation}
    \item \textbf{Energy Computation}: The energy of a new transaction is calculated
        using the inferred statistical model:
    \begin{equation}
        H(a_1, ..., a_N) = - \sum{i,j | i<j} e_{ij}(a_i, a_j) - \sum_i h_i(a_i)
    \end{equation}
    \item \textbf{Thresholding for Anomaly Detection}: A threshold $ T $ is defined
        based on statistical properties of known benign transactions. Transactions
        with $ H > T $ are flagged as suspicious.
    \item \textbf{Model Refinement}: The model is periodically updated by incorporating
        new benign transaction data, ensuring adaptability to evolving transaction
        patterns.
\end{enumerate}


% section 5.3 - applying efc on elliptic dataset
\section{Aplying EFC on Elliptic Dataset}
TBD

