\section{How EFC Works}
EFC was initially developed for Network Intrusion Detection Systems (NIDS) to
recognize both known and unknown attacks. The classifier is built upon statistical
physics principles, particularly the Potts model, which estimates probability
distributions of normal and anomalous network flows. The primary innovation of
EFC is its ability to generalize well while maintaining computational efficiency,
making it an attractive option for AML applications where real-time detection of
suspicious transactions is crucial \cite{souza2022}.

EFC works by analyzing transaction data and giving each transaction a score based
on how different it is from normal patterns. If a transaction's score is too high,
it is flagged as suspicious. The system considers two key factors:

\begin{itemize}
    \item \textbf{Individual Transaction Features:} These help determine whether
    a transaction follows normal patterns.
    \item \textbf{Relationships Between Features:} This helps detect complex fraud
    techniques where multiple factors interact.
\end{itemize}

When the system detects a transaction that doesn't match any known pattern, it
marks it as suspicious and sends it for further review.
