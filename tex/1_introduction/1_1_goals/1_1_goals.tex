\section{Goals}
\textit{\textbf{Research Question:} To what extent is the Energy-based Flow Classifier (EFC) a viable and effective alternative
to conventional machine learning models for detecting illicit Bitcoin transactions given the real-world challenge of label scarcity?}

\begin{itemize}
    \item \textbf{Main Objective }{To empirically evaluate the effectiveness of the Energy Flow Classifier (EFC), a one-class anomaly
    detection model, for identifying illicit Bitcoin transactions in the Elliptic dataset, particularly under the real-world
    constraint of label scarcity.}
    \item[] \textbf{Specific Objectives:}
    \begin{itemize}
        \item To adapt and apply the Energy Flow Classifier (EFC), a model originally from network intrusion detection,
        to the domain of cryptocurrency fraud, training it exclusively on licit transaction data to establish a baseline
        performance on the raw, imbalanced Elliptic dataset.
        \item To systematically investigate the impact of various data balancing techniques including undersampling, oversampling,
        and the Synthetic Minority Over-sampling Technique (SMOTE) on the EFC's ability to classify illicit transactions.
        \item To analyze the effect of dimensionality reduction on EFC's performance by applying feature selection (SelectKBest)
        to identify an optimal subset of features for distinguishing between licit and illicit activity.
        \item To assess the performance of a combined strategy that integrates feature selection with data balancing (SMOTE)
        to determine if this synergistic approach yields superior classification results on a realistic, imbalanced test set.
        \item To compare the performance of the optimized EFC configuration against established methodologies from previous
        work, providing a conclusive answer on its viability as a practical tool for fraud detection in label-scarce cryptocurrency
        environments.
    \end{itemize}
\end{itemize}
