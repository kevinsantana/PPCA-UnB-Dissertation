\section{Context}
What is the significance of money in modern economic systems? Can it be described merely as a medium of exchange, or
does it represent something more fundamental? When engaging in transactions on the open market, how important is the
role of money in determining the value of goods and services exchanged? Is the definition of money as a medium of
exchange sufficient to fully understand its role in contemporary economic systems? The conceptualization of money as an
embodiment of value is crucial in understanding the mechanisms behind wealth accumulation. Wealth creation is not
arbitrary, but rather reflects the perception of value generated by individuals or entities in society. The evolution of
human civilization has been driven by the development of sophisticated systems for storing and exchanging value, which
have facilitated economic transactions and enabled the emergence of complex socio-economic structures.

The banking system operates through a process of fractional reserve lending, whereby only a small percentage of deposits
are held in custody, while the remaining amount is loaned out to other individuals or businesses \cite{diamond1983bank}.
In this scenario, if an individual has a balance of 100 in their account, the bank may only hold 3 in custody, and the
remaining 97 may be lent out to Pablo to purchase goods or services. The digits on the screen increase without any link
to anything of intrinsic value, such as gold or other tangible assets. This new money is created through the promise of
payment from Pablo, which constitutes a debt obligation. This process of creating money through debt is a form of
modern-day slavery, whereby the life energy of the population is siphoned off by a parasitic mechanism. When Pablo
spends the 97 in an establishment, the owner will deposit that money in another bank, perpetuating the cycle of debt.
The original 100 can be multiplied indefinitely through this process, with more than 90\% of the money in circulation
consisting of digital entries on a computer screen with no intrinsic value attached. In essence, all magic money created
is collective debt, and the confidence placed in this system is what enables it to continue functioning. However, if
this confidence  were to be lost, the entire financial system would collapse, leaving individuals and businesses without
access to the means of exchange necessary for economic activity \cite{minsky2008stabilizing}.

In 2008, a global financial crisis in the real estate sector occurred \cite{bordo2008historical}, which was caused by
the State's Magic money machine providing easy credit \cite{murphy2008analysis}. A cryptographic document began
circulating on a mailing list for cryptographers, signed by the pseudonymous Satoshi Nakamoto, which compiled detailed
data and discoveries made through cypherpunk innovations. Nakamoto utilized this knowledge to create an electronic
transaction system that did not require the involvement of a third-party moderator. In essence, Nakamoto's work involved
using math, programming, and cutting-edge cryptography to publish a map for removing governmental presence from
financial transactions. The recent collapse of the economy had demonstrated that governments cannot be trusted, and
Nakamoto's solution was to create a currency that was mathematically impossible to be corrupted --- Bitcoin \cite{nakamoto2008bitcoin}.

Although Bitcoin does present a solution to the corruptible nature of money, it does possess flaws and is target of more
sofisticated frauds. Such as \textit{Spoofing} and \textit{Layering}.