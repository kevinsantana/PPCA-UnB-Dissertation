\subsection{Blockchain}
A distributed ledger system is composed of multiple nodes that broadcast transactions. To ensure consensus on the correct ledger, it is
necessary to develop a mechanism that allows all nodes to agree on the validity of each transaction~\cite{el2018review}.

The core concept of the original Bitcoin paper~\cite{nakamoto2008bitcoin} is based on the assumption that a distributed ledger will be
trusted if it has been subject to a large amount of computational effort. This idea is implemented through the use of the many-zeroes
game, which involves proving that a particular block in the chain contains a hash that is difficult to reverse-engineer.

Rather than hashing the entire ledger repeatedly, it is more efficient to allow for the accumulation of computational effort over time.
Transactions are grouped into blocks and added to the chain in a linear fashion, with each new block containing a reference to the
previous one. This approach allows for the creation of a tamper-evident history of transactions that is trusted by network participants
due to the large amount of computational work required to manipulate it.

%% figure 1.11
\figuraBib{img/chapter-1/1_2_9_blockchain/11_blocks.png}{Blocks on a blockchain}{3blue1brown_blocks}
{blocks}{width=.85\textwidth}%

The block is a collection of transactions enclosed with a unique identifier, known as proof-of-work (PoW), which serves as evidence of
the computational effort expended in validating the block. In PoW schemes, the miner must solve a complex mathematical problem to validate
the block and add it to the blockchain. The difficulty level of this problem is determined by the target number of leading zeros required
in the hash value of the block.

A block is considered valid if it contains a proof-of-work (PoW) value, analogous to how a transaction is only considered valid when signed
by its sender. Additionally, maintaining the integrity of the blockchain requires that blocks are not rearranged as this would disrupt the
transaction history. To address this issue, each new block must begin with the hash of the previous block (hash-based chain), ensuring that
the order of the blocks remains consistent.

%% figure 1.12
\figuraBib{img/chapter-1/1_2_9_blockchain/12_block-ordering.png}{Because blocks are chained together like this,
instead of calling it a ledger, this is commonly called a “blockchain”}{3blue1brown_blockchain}{blockchain}{width=.85\textwidth}%
