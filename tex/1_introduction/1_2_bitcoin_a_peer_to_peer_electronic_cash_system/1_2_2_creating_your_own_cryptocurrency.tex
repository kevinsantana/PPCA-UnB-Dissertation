\subsubsection{Creating Your Own Cryptocurrency}
One common scenario where distributed ledgers can be useful is when multiple individuals frequently exchange small amounts
of money, such as paying for shared expenses like dinner bills. To simplify this process, they may choose to maintain a
communal ledger that records these transactions in a manner similar to using physical currency. By doing so, participants
can easily keep track of their contributions and settle up when necessary.

%% figure 1.1
\figuraBib{img/chapter-1/1_2_2_creating_your_own_cryptocurrency/1_ledger.png}{A ledger is a record of financial transactions,
utilized for monitoring the accounts of all parties involved}{3blue1brown_ledger}{ledgerr}{width=.85\textwidth}%

The proposed ledger system would be a publicly accessible platform similar to a website where users can add new entries.
At the end of each month, participants could review the list of transactions and calculate the total sum. If an individual
has spent more than they have received, they would contribute that amount to the collective pool, while those who have
received more than they have spent would withdraw funds from the pool.

The protocol for participation in the system involves the following steps:

\begin{enumerate}
    \item Any individual can add entries to the distributed ledger;
    \item At the end of each month, all participants gather to reconcile their accounts using physical currency.
\end{enumerate}

However, a potential issue arises with a public ledger that allows any individual to add entries. How can one ensure that
Bob does not enter ``Alice pays Bob 100`` without Alice's approval? There is a Cryptography solution: \emph{Digital signatures}.
