\subsubsection{Distributing The Ledger}
The distributed nature of the blockchain technology used by the ledger system necessitates the use of a centralized platform for public access
and modification of the ledger's contents. However, this raises concerns regarding the trustworthiness of the entity responsible for hosting
the website and regulating the rules governing the addition of new entries to the ledger. In particular, it is important to identify and
evaluate the credibility of the entity that controls the website and establishes the protocols for updating the ledger.

In order to eliminate trust in a centralized system where one ledger is maintained, we will replace this with a decentralized approach, where
each individual will maintain their own copy of the ledger. This will enable transactions, such as "Alice pays Bob 100 LD," to be broadcasted
and recorded on personal ledgers by all parties involved in the network.

The distributed ledger technology employed by Bitcoin involves the broadcast of transactions by users, which are then recorded on a
decentralized set of records. This eliminates the need for trust in a central authority. However, this system is problematic due to the
possibility of disagreement among participants regarding the correct ledger. For example, when Bob receives a transaction "Alice pays Bob
10 LD", how can he be certain that everyone else has received and believes in the same transaction? If even one person does not know about
this transaction, they may not allow Bob to spend those 10 Ledger Dollars later

%% figure 1.9
\figuraBib{img/chapter-1/1_2_6_distributing_the_ledger/9_are-these-the-same.png}{If everyone keeps a unique copy of the ledger, how
can we ensure that everybody agrees on what it should say?}{3blue1brown_ledgers}{ledgerr}{width=.85\textwidth}%

The verification of the integrity and consensus of a blockchain network relies on a distributed ledger system where all participants maintain
a copy of the same transaction history. The trustworthiness of this system is predicated on the assumption that all nodes will accurately
record and remember past transactions, which may be subject to potential inconsistencies or discrepancies in the event of faulty or malicious
behavior. Therefore, it is essential to establish a mechanism for ensuring that the distributed ledger remains consistent across all
participating nodes. The solution proposed by Satoshi Nakamoto in 2008 for decentralized systems was a method to validate the validity of
a growing document, such as a ledger, without relying on a central authority. This problem was solved through the use of computational
work to determine trustworthiness, where the ledger with the most computational effort invested in it is considered legitimate. The idea
is that if an individual attempts to manipulate the ledger, it would require an impractical amount of computational power, making fraudulent
transactions computationally infeasible. This concept forms the core of Bitcoin and other cryptocurrencies.
