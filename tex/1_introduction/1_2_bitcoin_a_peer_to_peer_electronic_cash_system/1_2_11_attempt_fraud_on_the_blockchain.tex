\subsubsection{Attempt Fraud On The Blockchain}

To evaluate the trustworthiness of this method, it is instructive to consider what steps an individual, such as Alice,
would need to take in order to deceive the system. In particular, suppose that Alice desires to purchase an item from Bob
for 100 Ledger Dollars (LD), but does not actually possess those LDs. She might attempt to send a block to Bob containing
a line indicating "Alice pays Bob 100 LD" without broadcasting this block to the broader network. By doing so, Bob would
believe that he had been paid and provide Alice with the item she desires. However, at a later time, Alice could re-enter
the economy and spend those same 100 LD elsewhere. When Bob attempts to spending those same 100 LD, other individuals in
the network may not recognize them as valid, leading to the potential for deception to be detected.

The process of creating a fraudulent transaction in a blockchain network requires a valid proof-of-work (PoW) that is found
before other miners who are listening to the same set of transactions as the attacker, each working on their own block.
This is a difficult task but can be accomplished if the attacker has a significant portion of the network's computation
power. If Alice is able to find the PoW before other miners, she can create a fraudulent transaction and present it to
Bob (but not to anyone else)~\cite{fang2022cryptocurrency}.

However, Bob will continue to receive broadcasts from other miners, and Alice did not inform these miners about the block
she produced for Bob. Therefore, they will not include this block in their own versions of the blockchain. As a result,
Bob will be hearing conflicting chains: one from Alice and another from everyone else~\cite{TAN2022101625}. According to
the protocol, Bob always trusts the longest chain he knows about, which may create challenges for detecting and resolving
fraudulent transactions in the network.

The probability of Alice's computational resources being smaller than the combined computational resources of the rest of
the network is high, and as a result, it is more likely for the rest of the network to find a valid proof of work for their
next block before she does. Additionally, if Alice has less than 50\% of the total computation on the network (which is
highly probable), she will outpace everyone else indefinitely will be nearly impossible~\cite{nakamoto2008bitcoin}.

Eventually, when Alice fails to maintain her chain longer than the rest of the network, Bob will reject what he is hearing
from Alice and follow the longer chain that everyone else is working on. This is because creating blocks requires significant
computational effort, making it extremely difficult for any individual or group to manipulate the consensus~\cite{szabo2005bit}.

It's worth noting that while building a single fraudulent block may be possible, maintaining the lie for an extended period
is challenging. Therefore, users should exercise caution and wait for several new blocks to be added on top of a newly
discovered block before trusting it as part of the main chain. By doing so, they can ensure that they are not being tricked
by a malicious actor attempting to manipulate the network~\cite{dupont2019cryptocurrencies}.

%% figure 1.14
\figuraBib{img/chapter-1/1_2_11_attempt_fraud_on_the_blockchain/14_dont-trust-yet.png}
{Blocks are most trustworthy when they aren't brand new}{3blue1brown_trust}{trust}{width=.85\textwidth}%
