\section{Justification}
According to data from Mercado Bitcoin, the platform has facilitated a total trading volume of 50 billion reais since
its inception in 2013, with 200 assets and 3.8 million clients actively trading on the platform. Bitcoin represents 40\%
of the \$1 trillion in crypto assets outstanding. Although is not possible to measure the exact amount of volume target
of scams, fraud or even market manipulation is imperative to have a mechanism to identify these anomalies.

Without a proper mechanism to identify or even prevent such fraudulent actions, can deteriorate public trust into the
exchange leading to loss on revenue and market share. On the legal side, the cryptocurrency exchange can be prosecuted
by traders, companies and other individuals trading on the platform. This also applies to Goverment and authorities.
Futhermore, the inaction toward these anomalies can hold back the cryptocurrency economy, preventing new features and
the development of the web3 economy itself.

While these concerns are not unique to cryptocurrency exchanges, it is imperative to have means to detect and prevent
fraudulent actions in any exchange, including those involving stocks and other traditional financial assets. Despite the
importance of this matter, the current measures to detect and stop such actions are often rudimentary and limited, with
nonexistent or insufficient regulation being a common obstacle.

All this effort could benefit brazilian authorities, specially \textit{Comissão de Valores Mobiliários} (CVM),
\textit{Banco Central do Brasil} (BACEN). Finally the recently passed project on the regulation of cryptocurrency assets
by the \textit{Câmara dos Deputados} PL 4401/2021, which brings more regulation to the sector.
