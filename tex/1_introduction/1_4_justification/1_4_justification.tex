\section{Justification}
According to data from Mercado Bitcoin, the platform has facilitated a total trading volume of 50 billion since its inception in 2013, with 200 
assets and 3.8 million clients actively trading on the platform. Bitcoin represents 40\% of the \$1 trillion in crypto assets outstanding.

The limited regulatory oversight of cryptocurrency trading has raised concerns among authorities regarding the susceptibility of virtual currencies 
to fraudulent activities. This concern is due to factors such as skepticism about the effectiveness of exchange's efforts to prevent cheating, wild 
price swings that could facilitate manipulation, and the lack of regulations comparable to those governing traditional financial assets. While 
these concerns are not unique to cryptocurrency exchanges, it is imperative to have means to detect and prevent fraudulent actions in any exchange, 
including those involving stocks and other traditional financial assets. Despite the importance of this matter, the current measures to detect and 
stop such actions are often rudimentary and limited, with nonexistent or insufficient regulation being a common obstacle.

All this effort could benefit brazilian authorities, specially \textit{Comissão de Valores Mobiliários} (CVM), \textit{Banco Central do Brasil} 
(BACEN). Finally the recently passed project on the regulation of cryptocurrency assets by the \textit{Câmara dos Deputados} PL 4401/2021, which 
brings more regulation to the sector.
