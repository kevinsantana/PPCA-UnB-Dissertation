\section{Money is Corruptible}
Bitcoin (BTC) emerged on the scene in late 2008, allegedly as a response to the financial crisis of 2007-2008, and some suggest that it was also motivated by frustrations with the bureaucratic nature of the Japanese banking system. However, the latter claim ventures into more conspiratorial territory, largely based on the notion that the original author or authors of the Bitcoin whitepaper may have had connections to Japan \cite{nakamoto2008bitcoin}. Nevertheless, prior to delving into the intricacies of Bitcoin, it is crucial to first explore the concept of money and, more significantly, its foundational aspect: value.

What is Money? Or rather, what does money represent?

\subsection{What is Money}
If we asked the question: \textit{What is man's greatest invention?} What would your answer be? There's a lot of options. 
Would it be fire? Because it gives us warmth, protection, and the ability to cook our meals? Or perhaps you would pick the wheel? Because it's the driving
force being the beginnings of trade, commerce, and travel. While both of those are excellent choices, most of the time when we think about the greatest inventions of mankind, we tend to forget one of the most important ones of all: money. Unlike tangible inventions such as fire and the wheel, money possesses an immaterial nature. It exists as a conceptual construct, lacking inherent value, and its significance is derived solely from the subjective importance we attribute to it. This intangible nature of money often distinguishes it from other notable inventions in the collective human consciousness. \cite{smith2010wealth}.

Notwithstanding the illusory nature of money, its significance remains unaffected. Prior to the establishment of monetary systems, human societies engaged
in direct exchange of goods and services known as the Barter system. In this system, individuals traded commodities without an assigned intrinsic value,
relying solely on subjective evaluations of desired items. Consequently, each transaction was contingent upon the willingness of parties involved to forfeit
possessions in pursuit of their desired commodities. Such an exchange mechanism resembled a game-like scenario \cite{durlauf2016new}. If a situation arose where I desired vegetables for my meal, but my only possession was cattle, I would be obliged to offer one of my animals in exchange for bags of vegetables. Similarly, if I required footwear but specialized in tent production, I would have to surrender an entire tent to obtain a pair of slippers. This barter-based system reveals a prominent issue known as asymmetry. As a tent-maker, the exchange of an entire dwelling for simple footwear would undoubtedly leave me feeling disadvantaged. The absence of a standardized medium of exchange presented significant challenges in facilitating agreements between individuals with disparate needs. Moreover, the reliance on a fortuitous occurrence of complementary wants, wherein two individuals simultaneously sought the reciprocal possession, further complicated matters, rendering the process inefficient \cite{goodhart1998two}.

Our monetary system serves not only as a medium of exchange but also as a store of value. However, prior to the advent of money, certain individuals were
unable to effectively preserve their wealth through no fault of their own. Consider the scenario of a farmer selling tomatoes and a tent maker. The tent
maker has the ability to amass a substantial portfolio of real estate in the form of tents, which can be bartered year-round with individuals in need of
shelter. Consequently, the tent maker has the opportunity to accumulate wealth. In contrast, the farmer selling tomatoes can only engage in barter
transactions during the tomato season. Moreover, due to the perishable nature of tomatoes, long-term storage is not feasible. Thus, despite exerting
comparable efforts in their respective businesses, the farmer had no viable means to sustain wealth throughout the year \cite{de2016origins}. There's also the problem of having something that only very few people want. Nowadays, when starting a business, you're often told to find a niche. A small group of people who are very interested in what you have to offer. Before money was a thing, that advice would have left you with nothing worth bartering.

In societies where possessions of high demand, such as weapons, animal skins, and salt, held significant value, individuals who possessed such commodities
acquired substantial wealth. The awareness that these items were universally sought-after prompted individuals to engage in anticipatory buying, even if
immediate need was absent, to secure future trading opportunities. As a consequence, the emergence of commodity money ensued, whereby goods and services
were exchanged for commonly recognized items such as salt or weapons, facilitating subsequent transactions with other parties \cite{polanyi1965trade}.

Humanity progressed beyond direct barter, encompassing a diverse range of commodities including salt, weapons, and minute collectibles like shells and beads.
This evolution introduced a more efficient method of trade and exchange. Rather than directly swapping goods and services, individuals adopted the practice
of using arbitrary objects as intermediary placeholders of value, effectively functioning as IOUs (I Owe You). Subsequently, these placeholders could be
utilized to acquire desired goods and services from others. This concept proved remarkably ingenious, ultimately leading to a global transition from the
Barter system to the monetary exchange system \cite{graeber2012debt}. However, there has been a persistent limitation associated with this form of exchange. In order for currency to exhibit intrinsic value, it requires a degree of scarcity \cite{smith2010wealth, ricardo1821principles}. The more easily accessible an item is, the lower its perceived worth \cite{marshall2009principles}. When an item is readily obtainable by anyone, its value diminishes considerably. As a result, substances such as sand or shells, which can be effortlessly collected from any beach, do not effectively function as indicators of value \cite{principlesmenger, hicks1936keynes}.

In approximately 770 BC, China witnessed the emergence of the earliest metal coins, marking a significant milestone in the evolution of currency. As a
tribute to their historical currency systems, the Chinese craftsmen ingeniously crafted miniature replicas of tools that were previously utilized as
forms of exchange. With the intention of ensuring convenient handling, the coins were deliberately designed in a circular shape, allowing easy retrieval
from pockets without causing any discomfort to the fingers. These coins were predominantly cast using bronze, thereby bestowing intrinsic value upon them.
This transition marked a pivotal moment in history, as money transformed from a mere symbol to a tangible entity of worth. The scarcity of bronze, a
resource not readily available on any beach, further amplified the significance of these coins \cite{li2003, hartill2005}. During this period, the concept of money had not yet deviated from material reality. The valuation of a coin corresponded directly to the intrinsic value of the metal constituting the coin. For instance, a coin crafted from 1 gram of gold possessed an equivalent worth of precisely 1 gram of gold. This quantifiable attribute allowed for straightforward verification through direct measurement, enabling individuals to visually ascertain that the coin indeed comprised 1 gram of gold.

The realization of the potential power of money was swift among Kings and Rulers, as highlighted by Cribb \cite{cribb1991}. This understanding led to the
creation of the first official money mint by Alyattes, the King of Lydia, around 600 BC. These coins were minted from a blend of silver and gold, with
each coin featuring a distinctive image serving as a denomination. Consequently, individuals could effortlessly determine the value of their metal
possession by observing the pictorial representation on the coin's surface \cite{deVries2008}. The pursuit of greater wealth among Kings led to the devaluation of coins through the reduction of precious metal content and the inclusion of cheaper metals, as highlighted by Weatherford \cite{weatherford1997}. This resulted in the divergence between the face value and actual worth of circulating coins, establishing the illusion of money. The value of coins became divorced from the intrinsic value of their metal composition, relying instead on the dictates of rulers and financial institutions \cite{ferguson2009}. As an example, the British Pound Sterling ceased to represent a fixed quantity of Sterling Silver and instead denoted a unit of currency determined by authoritative decree.

The emergence of international trade exposed the impracticality of metal coins, leading to the introduction of IOU certificates by Kings to facilitate
long-distance transactions, as discussed by Graeber \cite{graeber2011}. These certificates, bearing the King's stamp, gained trust and were believed
to hold value, as they were expected to be exchangeable for equivalent coins. Initially, this belief corresponded to reality. With the proliferation of IOU certificates in circulation, the necessity for physical coins diminished. Ultimately, the value of the certificates became divorced from their direct convertibility into gold and silver coins. Instead, their value relied on collective trust and shared belief, as highlighted by Ingham \cite{ingham2004}. This shift allowed the paper certificates to retain value based on our perception, even in the absence of immediate exchange for tangible precious metals.

\subsection{The Illusion of Money}
From Ancient Kings to modern day governments and Central Banks, money has remained an illusion. A mere representation whose value is determined by
the importance people place on it.

The ten thousand Singapore Dollars banknote, while no longer in production, remains the highest denomination in circulation \cite{goodhart1998}.
Despite its intrinsic production cost of less than 20 cents, the value of this paper note is upheld by the illusion perpetuated by the fiat
currency system \cite{gupta2019}. Presently, its monetary equivalence to seven thousand three hundred and forty-five US Dollars enables its
utilization in acquiring substantial assets such as houses, cars, and even valuable commodities like gold.

“Fiat” is the fancy word we use to describe the modern-day illusion. It's a Latin word that translates to “let it be done.” It's a decree by the
government that, in the case of money, determines what its value is and enforces it as legal tender \cite{reinhart2018, friedman2000}.

The elusive nature of money often evades careful consideration, yet akin to historical rulers, contemporary governments possess an understanding of the influential power of currency and persistently strive for its accumulation, as expounded by Graeber \cite{graeber2011}. Recognizing that the possession of greater quantities of these paper instruments equates to amplified authority, governments adopt the approach of generating additional currency ex nihilo, as highlighted by Mankiw \cite{mankiw2014}. For instance, in the scenario where the United States government necessitates \$340 million dollars to procure an F-22 jet, it possesses the capacity to create the required funds through the act of monetary printing.

But there is one problem with this: inflation.

The fundamental attribute of money lies in its role as a medium of exchange, conferring value upon it, as emphasized by Mankiw \cite{mankiw2014}. Consequently, the quantity of money in circulation should align with the aggregate production of goods and services. Should the issuance of money exceed the availability of goods and services, with all else remaining constant, the resultant effect is an escalation in prices and a subsequent devaluation of the currency itself. This concern resonates with economists and the general population, including individuals such as ourselves, as underscored by Blinder \cite{blinder2010}, particularly in the context of the current global reserve currency, the United States Dollar.

The year 2020 proved to be an exceedingly challenging period for the world at large, as the onset of the pandemic necessitated the temporary closure of numerous economies, resulting in a considerable reduction in the availability of goods and services and a marked decline in overall economic output, as outlined in the World Economic Outlook report by the International Monetary Fund \cite{imf2020}. In an effort to avert economic collapse and the potential disintegration of societal systems, the US government embarked on an unprecedented scale of monetary expansion, surpassing any previous instances of currency printing in its history \cite{blinder2020}. As in 2021, the current state of affairs reveals a considerable expansion of the US dollar supply, with approximately 40\% of the existing currency having been printed within the last 18 months \cite{fedmoneysupply}. This substantial increase in money supply in relation to the country's output has raised concerns regarding the potential for significant price inflation \cite{Blanchard2021}. Observable evidence of this trend is already apparent in the substantial rise in commodity prices, such as the tripling of lumber prices compared to a year ago. Additionally, discernible price increases can be observed in everyday experiences, including slight increments in prices at favorite restaurants, such as a modest 20-cent rise in the cost of guacamole at Chipotle \cite{BLS}. Although the provision of stimulus and unemployment checks by governments to their citizens may initially appear beneficial, it entails a double-edged sword. While it undoubtedly assists individuals in dire economic circumstances, it also introduces challenges. Presently, the combined factors of inflationary pressures and economic slowdown have created difficulties for individuals seeking suitable employment opportunities, not solely due to a lack of willingness, but also because certain job options may be less desirable than available alternatives \cite{cbo2020, kahn2020}.

An illustrative example can be observed in the United States, where the law does not mandate a minimum wage for individuals working as waiters or waitresses \cite{dol}. Consequently, some employees in these roles receive meager hourly wages, such as \$2 to \$3, with tips constituting a substantial portion of their earnings. However, due to the implementation of various restrictions and regulations nationwide, coupled with a decrease in customer traffic, there has been a reduction in both customer volume and disposable income, thereby leading to a decline in tip revenue \cite{bls2022}. Inadequate income for employees may result in higher turnover rates as financial needs are not being met. This situation poses a significant risk to businesses, as the lack of a sufficient workforce can ultimately lead to business closure, setting in motion a cascading effect \cite{azar2020}. A valid concern arises regarding the motivation to actively seek employment when the potential income from unemployment and stimulus checks surpasses that from being employed. This circumstance prompts an examination of available options. Notably, the Federal Reserve of the United States employs a strategic approach to injecting funds into the economy, a process that may not be widely acknowledged, thus stimulating economic activity without substantial public scrutiny \cite{frb}. Consequently, the relative attractiveness of alternative income sources may influence individuals' decision-making regarding employment prospects \cite{cbo2021}.

The United States had accumulated a staggering national debt of \$29 trillion prior to 2020, an astounding and challenging figure to comprehend \cite{usdebt}. This debt is primarily financed through the issuance of bonds and treasury notes, which are essentially contractual instruments offering repayment of a predetermined principal sum alongside interest \cite{treasurysecurities}. Presently, investing in a 10-year U.S. Treasury bond would yield a modest return of 1.23\% upon maturity. Therefore, investing \$1,000 today would result in a nominal return of a mere \$12.30 by 2031. However, this return fails to keep pace with the targeted inflation rate, projected to be around 2\% annually \cite{frbinflation}. It should be noted that actual inflation rates may surpass the target, although that discussion is beyond the scope of the current context. Consequently, investing in government notes issued by one's own country, whose currency is utilized in daily transactions, leads to a gradual erosion of purchasing power over the course of a decade. Irrespective of these concerns, financial institutions, businesses, and individuals worldwide participate in the acquisition of bonds and treasury notes, thereby providing governments with discretionary funds for utilization \cite{treasurysecurities}. However, when the government confronts the need to fulfill its debt obligations, the previously obtained funds have been fully expended. Consequently, the government initiates repurchases of treasuries and bonds, confining such transactions to prominent financial institutions and remunerating them by means of freshly created money, effectively conjured from nothingness. The Federal Reserve, for instance, has repurchased over \$1 trillion in bonds since March 2020, with plans to persist in such actions well into the future \cite{federalreserve}.

By virtue of government injections, banks are empowered to expand their lending activities, thereby increasing interest income and fostering economic growth \cite{federalreserve}. However, this surge in lending concurrently expands the aggregate money supply, leading to a depreciation in the value of each individual dollar. The implementation of multi-trillion dollar stimulus payments and infrastructure packages raises questions regarding the sustainability of such practices. The influx of new money results in a devaluation of existing money, whereby the balance in an individual's bank account remains unchanged, yet its purchasing power diminishes owing to the influx of newly minted money \cite{currencydevaluation}. Consequently, the retention of wealth in a fiat currency like the US dollar progressively erodes its value, ultimately impeding the ability to acquire goods and services despite nominal bank balances.

The reality that money is nothing but an illusion is one that we must all embrace. Because only then will the path to financial freedom become
clearer. Understanding that money does not have any intrinsic value in itself, but instead only inherits the value we give to it.

As the money supply continues to expand, the purchasing power of each individual dollar held in one's possession inevitably erodes, whereas the dollar-denominated value of global assets tends to appreciate \cite{moneyprinting}. Nevertheless, this perceived growth can be likened to an optical illusion, employing deceptive mechanisms. Despite the seemingly unrelenting ascent of the stock market, the underlying reality is far from reassuring. The relentless depreciation of the currency compounds the situation, eroding its value on a daily basis. For example, if the Dow Jones Industrial Average, which serves as a benchmark for the performance of 30 major US companies, were denominated in terms of gold rather than USD, it becomes apparent that its value has essentially stagnated since 1997 \cite{stockmarketillusion}.

But what's the end goal of all of this? With fiat and an unlimited supply of money, will the value of each currency just continue to decrease until
the end of time? Will the gap between the rich and the poor continue to grow wider? Or are we going to finally fix a problem as old as man itself, and
stop placing our financial success in the hands of those who are destroying it day by day? Money is corruptible.

Only time will tell, but just to know, there is a way out: \textbf{Bitcoin}.

\section{Bitcoin: A Peer-to-Peer Electronic Cash System}
Bitcoin is a decentralized digital currency that operates on a peer-to-peer network called the blockchain. It was introduced in a 2008 whitepaper by an anonymous person or group of people using the pseudonym Satoshi Nakamoto \cite{nakamoto2008bitcoin}. Bitcoin is not controlled by any central authority, such as a government or financial institution, making it a unique form of currency. It relies on cryptographic techniques to secure transactions and control the creation of new units.

In the subsequent sections, we will delve into the fundamental principles of Bitcoin, its architectural design, the blockchain technology underlying it, the representation of assets in the stock market, instances of fraudulent activities and scams, and ultimately explore potential solutions to address the inherent flaws associated with money and its susceptibility to corruption.

\subsection{Basic Concepts of Bitcoin}
At its core, Bitcoin is a form of digital money that can be sent from one party to another without the need for intermediaries like banks. Transactions are verified by network nodes through cryptography, and these verified transactions are bundled into blocks, forming a chronological chain of blocks, hence the term \textit{blockchain}.

This introduction aims to provide a concise overview of the key concepts that underpin the operation of Bitcoin. By delving into the blockchain technology, cryptographic security measures, consensus mechanisms, and its unique monetary policy, this study seeks to elucidate the scientific foundations of Bitcoin. Understanding these concepts is essential for comprehending the mechanisms driving the secure and transparent peer-to-peer transactions facilitated by Bitcoin. These concepts form the foundation of Bitcoin's innovative design and functionality, driving its potential to reshape financial systems and promote financial inclusion. Understanding these concepts is crucial for comprehending the underlying mechanisms of this groundbreaking cryptocurrency.

\begin{enumerate}
    \item Blockchain Technology: Bitcoin employs a distributed ledger known as the blockchain, which records all transactional data in a transparent and immutable manner. The blockchain consists of a chain of blocks, each containing a set of transactions, cryptographically linked to the previous block. This technology ensures the integrity and transparency of the transaction history \cite{nakamoto2008bitcoin} by employing cryptographic hashing algorithms \cite{merkle1988digital}.
    \item Cryptographic Security: Bitcoin utilizes cryptographic techniques to ensure the security and privacy of transactions. Public-key cryptography, specifically elliptic curve cryptography (ECC) \cite{johnson2001elliptic}, enables the creation of unique public and private key pairs for users. The private key is kept secret and used for signing transactions, while the corresponding public key allows for the verification of digital signatures \cite{dworkin2016recommendation}. Additionally, cryptographic hash functions, such as SHA-256 (National Institute of Standards and Technology, 2015), provide data integrity by generating fixed-size hash values, facilitating efficient verification of transactional data.
    \item Consensus Mechanisms: Bitcoin employs a consensus mechanism called Proof-of-Work (PoW) \cite{dwork1993pricing}, which ensures agreement among network participants on the validity of transactions and the ordering of blocks. Miners compete to solve computationally intensive puzzles, providing a probabilistic guarantee that a majority of the network's computing power is honest. This process incentivizes miners through the possibility of being rewarded with newly minted bitcoins and transaction fees \cite{nakamoto2008bitcoin}.
    \item Monetary Policy: Bitcoin's monetary policy is predetermined and governed by code, rather than central authorities. The total supply of bitcoins is limited to 21 million \cite{nakamoto2008bitcoin}. Through scheduled halvings, which reduce the block reward given to miners, the issuance of new bitcoins decreases over time until the maximum supply is reached. This deflationary feature distinguishes Bitcoin from traditional fiat currencies \cite{maurer2013}.
\end{enumerate}

On the upcoming sections, we explain in a more detailed way 

\subsection{Blockchain}
Blockchain is a decentralized and distributed digital ledger that employs cryptographic techniques to secure and maintain the integrity of transaction records across multiple computers or nodes \cite{nakamoto2008bitcoin}. It operates by grouping transactions into blocks and linking them together in a sequential and unalterable chain. The transparency and immutability of blockchain, coupled with consensus mechanisms for transaction validation, make it a promising technology with applications in various domains \cite{swan2015blockchain}.

One way to visualize a blockchain is as a digital ledger with multiple pages or "blocks" that are chained together. When a new transaction occurs, it is added to a new block in the chain, which is then verified by the network and added to the existing chain. This creates an immutable record of all transactions on the network, which can be accessed and audited by anyone with access to the blockchain.

Here's an example of how a blockchain might look:

\begin{figure}[h]
    \centering
    \includegraphics[width=0.6\textwidth]{blockchain_structure.png}
    \caption{Structure of a blockchain \cite{nakamoto2008bitcoin}}
    \label{fig:blockchain}
\end{figure}

```
Block 1 (hash: 0x123456789abcdef) -> Block 2 (hash: 0xdefcafbadcadabef) -> ... -> Block N (hash: 0xefghijklmnopqrst)
```
Each block contains a list of transactions, along with a timestamp and a unique cryptographic hash that identifies the block. When a new transaction is added to the network, it is broadcast to all nodes on the network, which then verify the transaction using consensus algorithms such as proof-of-work or proof-of-stake. Once the transaction is verified, it is added to a new block in the chain and the block is added to the existing chain.

Blockchain technology has numerous applications beyond cryptocurrency, including supply chain management, voting systems, and digital identity verification. It offers a secure and transparent way to store and manage data without the need for intermediaries or centralized authorities.
