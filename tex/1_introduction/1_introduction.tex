% chapter 1 - introduction
Money laundering is a high-impact global problem, with criminals laundering billions of dollars annually from serious felonies.
In recent years, cryptocurrencies have emerged as a significant channel for these illicit activities, largely due to the
pseudonymity they offer criminals. Machine Learning (ML) presents a powerful tool for identifying the complex patterns
associated with such illicit financial flows, potentially increasing detection rates while decreasing the high false-positive
rates common in traditional rule-based systems.

However, the practical application of ML in this domain faces a critical obstacle: the scarcity of labeled data. Supervised
learning algorithms, which typically offer high accuracy, are often unfeasible because large-scale labeled datasets of
illicit transactions do not exist. This label scarcity stems from the evolving complexity of money laundering
schemes, which makes identifying all illicit actors nearly impossible, and the fact that acquiring labels from law enforcement
or expert manual annotation is a costly and slow process.

Previous research exploring this problem on the Elliptic Bitcoin dataset has shown that traditional unsupervised anomaly
detection methods are also inadequate. A key finding is that illicit transactions do not necessarily present as statistical
outliers; instead, sophisticated criminals often attempt to mimic normal behavior, effectively hiding their activities
within clusters of legitimate transactions. While techniques like Active Learning have been shown to
achieve high performance by using a small fraction of labels, this dissertation explores a different approach
to contend with the challenge of label scarcity.

This research proposes the application of the Energy-based Flow Classifier (EFC), a novel algorithm initially developed
for network intrusion detection systems. The EFC is an anomaly-based classifier that uniquely infers its statistical model
based *solely* on benign (or licit) examples. By learning the fundamental properties and correlations
within normal traffic, it can classify new, unlabeled instances based on how much they deviate from the established benign
model.

The primary advantage of the EFC in the context of money laundering detection is its inherent ability to operate in an
environment with a severe lack of illicit labels, as it does not require malicious samples for training. Therefore, this
study aims to adapt and apply the EFC algorithm to the Elliptic dataset to evaluate its effectiveness in detecting illicit
Bitcoin transactions. We will assess whether the EFC provides a robust and practical alternative to previously studied
supervised and unsupervised methods, given its white-box nature and its design for label-scarce scenarios.

Bitcoin is an electronic transaction system that operates without a third-party moderator~\cite{nakamoto2008bitcoin}. It is
built on blockchain technology, where an immutable ledger of financial transactions is maintained through mathematics,
programming, and advanced cryptography. This distributed ledger architecture eliminates the need for central authorities
to establish trust. Although Bitcoin was designed to circumvent vulnerabilities in the traditional financial system
\cite{nakamoto2008bitcoin}, it is not immune to manipulation and anomalous activities and demands a robust detection
mechanisms~\cite{fang2022cryptocurrency, zhang2020financial,zainal2018review}. 

In fact, crypto-related fraud has become a significant threat, causing substantial financial losses and destroying
trust in the digital asset ecosystem. In 2023, for example, illicit addresses received \$24.2 billion in cryptocurrency,
indicating the scale of financial losses from scams, stolen funds, and other illicit activities \cite{chainalysis2024cryptocrime}.
These activities not only cause direct monetary damage to individuals and institutions, but also have broader implications,
such as undermining the legitimacy of cryptocurrency markets and hindering the widespread adoption of blockchain technology.
The need to develop effective methods for detecting and preventing cryptocurrency fraud is crucial to protect participants,
maintain market integrity, and ensure sustainable growth of the cryptocurrency industry \cite{scharfman2024, Khiari2025}.

However, detecting anomalous patterns within the intricate data streams of cryptocurrency transactions poses a significant
challenge. Like many modern datasets, these transactions are characterized by high dimensionality, evolving characteristics,
and substantial volume, which complicates the application of traditional anomaly detection methods. In this context, the
Energy-Based Flow Classifier (EFC) presents a promising approach rooted in statistical physics. Originally formulated using
the Inverse Potts model \cite{pontes2019}, the EFC characterizes the probability distribution of normal data flows through
an energy function derived from observed data patterns \cite{pontes2019}. Previous research has demonstrated the utility
of EFC in classifying unusual network traffic, suggesting its potential to adapt to detect fraudulent activity within cryptocurrency
systems \cite{pontes2019, souza2022novelopensetenergybased}. Its selection for this study is further motivated by its relatively
low computational complexity during both training and inference phases, rendering it a suitable candidate for real-time detection
applications where swift identification of illicit activities is paramount.

Building upon the promise of the Energy-Based Flow Classifier (EFC) framework, this paper presents a comprehensive empirical
evaluation of its application to the detection of illicit Bitcoin transactions. To this end, we first replicate a previous
study that employs machine learning algorithms such as K-Nearest Neighbors, One-Class Support Vector Machine, and Isolation
Forest for anomaly detection on the Elliptic dataset \footnote{Available at https://www.kaggle.com/ellipticco/elliptic-data-set}.
We then investigated the use of EFC as a potential alternative to these machine learning approaches, using the same data set
for consistency. Our findings confirm the EFC's ability to distinguish between licit and illicit transaction patterns based
on their energy profiles, showing strong performance in identifying illegal activity even when trained solely on licit data.
However, the results also highlight the critical sensitivity to specific configuration parameters. In particular, we observe
significant trade-offs between maximizing the detection rate of illicit transactions (recall) and minimizing false positives
(precision), especially concerning the energy threshold that defines anomalous behavior. 

Addressing the significant challenge of label scarcity inherent in datasets like Elliptic is crucial for developing effective
fraud detection systems. Traditional supervised machine learning methods often struggle in such scenarios due to the limited
availability of labeled illicit examples. This motivates the exploration of alternative approaches, particularly those capable
of learning from predominantly normal data. The Energy-based Flow Classifier (EFC) also emerges as a promising candidate
in this regard. Originally proposed for network intrusion detection \cite{pontes2019, souza2022novelopensetenergybased},
EFC was designed specifically to address key limitations of conventional ML classifiers, including the reliance on extensive
labeled datasets. A core strength highlighted in its foundational work is its ability to function as an anomaly-based classifier,
inferring a statistical model of normal behavior using only labeled *benign* (or licit, in our context) examples
\cite{souza2022novelopensetenergybased}. Deviations from this learned norm, characterized by higher 'energy' scores, are
then flagged as potential anomalies. This one-class learning paradigm directly tackles the label scarcity issue prevalent
in the Elliptic dataset, allowing us to model legitimate transaction patterns effectively even with few confirmed illicit
instances. Furthermore, EFC's demonstrated adaptability across different data distributions in network traffic analysis
suggests potential robustness in the dynamic environment of cryptocurrency transactions. Consequently, this paper evaluates
the suitability and performance of EFC for identifying illicit Bitcoin transactions by leveraging its capacity to model
normality from available licit data.

Our findings demonstrate that the Energy Flow Classifier (EFC) is a highly effective tool for this problem, especially when
its application is thoughtfully combined with data preprocessing strategies. We achieved a notable F1-Macro score of 0.770
by integrating the SelectKBest feature selection method with the SMOTE data balancing technique. The F1-Macro score was
chosen as the primary evaluation metric, following the precedent set by previous research on this dataset , and because
its methodology of averaging the F1-score for each class independently provides a balanced assessment that is crucial for
imbalanced datasets. This result highlights EFC's efficiency, suggesting it can be competitive with other sophisticated
approaches like active learning, by leveraging a one-class learning paradigm that is naturally suited for the core challenge
of label scarcity.

This dissertation is structured as follows to guide the reader through our investigation. Chapter 1 introduces the problem
of fraudulent cryptocurrency transactions, the research justification, and the context of our work. Chapter 2 provides the
necessary background on key machine learning techniques, the Energy Flow Classifier's mechanics, the cryptocurrency ecosystem,
and the Elliptic dataset. Chapter 3 reviews related work in the field. Chapter 4 details our study settings, including the
specific EFC configuration, data preprocessing steps, and the setup for our three main experiments. Chapter 5 presents the
empirical results of these experiments. Finally, Chapter 6 concludes the dissertation with a discussion of the findings,
a comparison of our work with previous studies, an acknowledgment of the threats to validity, and suggestions for future research.

% section 1.1 - goals
\section{Goals}
\textit{\textbf{Research Question:} To what extent is the Energy-based Flow Classifier (EFC) a viable and effective alternative
to conventional machine learning models for detecting illicit Bitcoin transactions given the real-world challenge of label scarcity?}

\begin{itemize}
    \item \textbf{Main Objective }{To empirically evaluate the effectiveness of the Energy Flow Classifier (EFC), a one-class anomaly
    detection model, for identifying illicit Bitcoin transactions in the Elliptic dataset, particularly under the real-world
    constraint of label scarcity.}
    \item[] \textbf{Specific Objectives:}
    \begin{itemize}
        \item To adapt and apply the Energy Flow Classifier (EFC), a model originally from network intrusion detection,
        to the domain of cryptocurrency fraud, training it exclusively on licit transaction data to establish a baseline
        performance on the raw, imbalanced Elliptic dataset.
        \item To systematically investigate the impact of various data balancing techniques including undersampling, oversampling,
        and the Synthetic Minority Over-sampling Technique (SMOTE) on the EFC's ability to classify illicit transactions.
        \item To analyze the effect of dimensionality reduction on EFC's performance by applying feature selection (SelectKBest)
        to identify an optimal subset of features for distinguishing between licit and illicit activity.
        \item To assess the performance of a combined strategy that integrates feature selection with data balancing (SMOTE)
        to determine if this synergistic approach yields superior classification results on a realistic, imbalanced test set.
        \item To compare the performance of the optimized EFC configuration against established methodologies from previous
        work, providing a conclusive answer on its viability as a practical tool for fraud detection in label-scarce cryptocurrency
        environments.
    \end{itemize}
\end{itemize}

