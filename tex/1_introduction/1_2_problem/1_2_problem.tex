\section{Problem}
\textit{Spoofing} and \textit{layering} are two types of financial fraud that are not unique to cryptocurrency and the
Bitcoin ecosystem. These scams have been extensively researched in traditional stock markets, but there is a lack of
deeper and more scientific research on their prevalence and impact in the cryptocurrency space. Such scams pose a
significant threat to cryptocurrency exchanges as they can manipulate prices, resulting in potential revenue losses and
damage to credibility. Due to the complex nature of these frauds, advanced Artificial Intelligence Models can be applied
to detect and prevent them \cite{zhang2020financial, zainal2018review}.

Cryptocurrency spoofing is the process by which criminals attempt to artificially influence the price of a digital
currency by creating fake orders. Spoofing is accomplished by creating the illusion of pessimism (or optimism) in the
market. Traders do this by placing large buy or sell orders without the intention of ever filling them. When investors
do this, they trick other investors into either buying or selling, and the price of the cryptocurrency stands the
possibility of being adjusted accordingly. The trader cancels the orders once the price of the cryptocurrency moves in
the direction they desire \cite{hasbrouck2013spoofing}.

Layering is another form of market manipulation where multiple non-bona fide orders are placed on one side of the order
book to create the illusion of market interest. Traders place and quickly cancel multiple orders at different price
levels, creating a stacked appearance of buy or sell interest. The intention is to attract genuine traders and influence
market sentiment \cite{brogaard2014market}.

Although these scams have been extensively researched in traditional stock markets, there is a lack of literature about
the use of ML techniques to detect and prevent them on cryptocurrency markets. Therefore, our initial goal is to analyze
the public trading data from Mercado Bitcoin using different Machine Learning techniques to detect anomalies that might
characterize \textit{spoofing} and \textit{layering} \cite{bhattacharyya2011survey}.