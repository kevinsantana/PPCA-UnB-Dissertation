\section{Problem}
\textit{Spoofing} and \textit{layering} are two types of financial fraud that are not unique to cryptocurrency and the Bitcoin ecosystem. These 
scams have been extensively researched in traditional stock markets, but there is a lack of deeper and more scientific research on their prevalence 
and impact in the cryptocurrency space. Such scams pose a significant threat to cryptocurrency exchanges as they can manipulate prices, resulting 
in potential revenue losses and damage to credibility. Due to the complex nature of these frauds, advanced Artificial Models can be applied to 
detect and prevent them.

Cryptocurrency spoofing is the process by which criminals attempt to artificially influence the price of a digital currency by creating fake 
orders. Spoofing is accomplished by creating the illusion of pessimism (or optimism) in the market. Traders do this by placing large buy or sell 
orders without the intention of ever filling them. When investors do this, they trick other investors into either buying or selling, and the price 
of the cryptocurrency stands the possibility of being adjusted accordingly. The trader cancels the orders once the price of the cryptocurrency 
moves in the direction they desire.

We plan to analyze the public trading data from cryptocurrency exchange Mercado Bitcoin, from Bitcoin market and build several Artificial 
Intelligence models to detect such anomalies.