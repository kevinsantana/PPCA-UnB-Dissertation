\section{Design Science Research Process}
The Design Science Research process consists of a series of iterative steps that guide the creation, validation, and
application of the designed artifact. The steps include problem identification and motivation, definition of objectives,
design and development, demonstration and evaluation, and communication of results \cite{hevner2010design,
peffers2007design}. In this study, these steps are adapted to the specific context of building a language model pipeline
for detecting anomalies on trading data of cryptocurrencies transactions.

\subsubsection{Problem Identification and Motivation}
The first step of the DSR process involves identifying and defining the problem to be addressed. In the realm of
cryptocurrency markets, anomalies are deceptive trading practices that can artificially inflate trading volumes and
manipulate market prices \cite{gandal2018price, cartea2016algorithmic}. The rapid growth of cryptocurrency markets,
including Bitcoin, has highlighted the need for effective tools to detect and prevent such manipulative activities
\cite{chan2017statistical}. Therefore, the problem addressed in this study is the detection of anomalies on
cryptocurrencies trading, using advanced language model techniques.

\subsubsection{Objectives}
The objectives of this methodology are as follows:

\begin{enumerate}
    \item To design a language model pipeline capable of analyzing Bitcoin time series data and identifying of possible
    patterns indicative of anomalies.
    \item To implement and develop the designed pipeline using state-of-the-art natural language processing (NLP) and
    machine learning (ML) techniques.
    \item To evaluate the performance of the language model pipeline using relevant metrics, such as precision, recall,
    F1-score, and receiver operating characteristic (ROC) curve analysis.
    \item To compare the effectiveness of the developed pipeline with existing methods for anomaly detection in
    cryptocurrency markets.
\end{enumerate}

\subsubsection{Design and Development}
The design and development phase encompasses the creation of the language model pipeline for detecting anomalies in the
cryptocurrencies time series dataset. The pipeline consists of the following components:

\begin{enumerate}
    \item Data Collection and Preprocessing: Historical Bitcoin trading data, including price, volume, and order book
    information, will be collected from Mercado Bitcoin public api. The data will be preprocessed to remove noise,
    normalize features, and create suitable input representations for the language model.
    \item Feature Engineering: Relevant features, such as price movement patterns, trading volume fluctuations, and
    order book imbalances, will be extracted from the preprocessed data. These features will serve as inputs for the
    language model.
    \item Language Model Architecture: A deep neural network-based language model, such as a transformer architecture
    \cite{vaswani2017attention}, will be designed to process the extracted features and learn patterns associated with
    fraudelent activities. Pre-trained models and fine-tuning will be evaluated as well.
    \item Training and Fine-Tuning: The language model will be trained using a labeled dataset containing instances of
    anomalies, as well as genuine trading behaviors. Fine-tuning techniques, such as transfer learning, will be employed
    to enhance the model's ability to detect manipulative activities \cite{howard2018universal}.
\end{enumerate}

\subsubsection{Demonstration and Evaluation}
The demonstration and evaluation phase assesses the effectiveness of the developed language model pipeline. The
following steps will be taken:

\begin{enumerate}
    \item Simulation and Testing: Simulated anomalies scenarios will be created to test the pipeline's ability to
    identify manipulative patterns. Additionally, the pipeline will be tested on a holdout dataset to evaluate its
    real-world performance.
    \item Performance Metrics: The pipeline's performance will be evaluated using standard metrics, including precision,
    recall, F1-score, and ROC curve analysis. These metrics will provide insights into the model's ability to correctly
    identify anomalies activities.
    \item Comparison with Existing Methods: The performance of the language model pipeline will be compared with
    existing methods for detecting anomalies in cryptocurrency markets. This comparison will highlight the pipeline's
    innovation and effectiveness.
\end{enumerate}

\subsubsection{Closure}
The final phase of the DSR process involves communicating the results of the study. This includes presenting the design
and development of the language model pipeline, detailing the evaluation outcomes, and discussing the implications of
the findings for detecting anomalies in Bitcoin trading.

Throughout the research process, ethical considerations related to data privacy, potential bias, and unintended
consequences of model deployment will be taken into account \cite{barocas2017fairness, mittelstadt2016ethics}. Measures
will be implemented to ensure the responsible use of technology in detecting manipulative activities.

It is important to acknowledge certain limitations of the proposed methodology, such as the availability and quality of
labeled training data, the potential for false positives and false negatives in the detection process, and the
generalizability of the language model to evolving trading patterns.

%% figure 4.1
\figuraBib{img/chapter-4/4_1_design_science_research_process/18_methodology.png}{Proposed
methodology}{}{methodology}{width=.85\textwidth}%